\section{Оформление .h файлов библиотеки}\label{file:h}
\subsection{Общее оформление}
\begin{enumerate}
	\item Файл должен включать в себя защиту от повторного включения в процесс компиляции по типу \textit{ifndef-define-endif}, оканчивающуюся пустой строкой.\\\textbf{Пример:}\begin{lstlisting}[language=C++, frame=tlBR, basicstyle=\fontsize{10}{10}\ttfamily]
#ifndef STM32F2_API_STM32_F20X_F21X_PORT_H_
#define STM32F2_API_STM32_F20X_F21X_PORT_H_
	
#endif
 \end{lstlisting}
	
	\item После \textit{define} строки защиты следует пустая строка, за которой располагается \textit{include} на файл конфигурации библиотеки, имеющий имя \textbf{perfix\_conf.h} (название библиотеки выступает в качестве префикса).\\\textbf{Пример:}\begin{lstlisting}[language=C++, frame=tlBR, basicstyle=\fontsize{10}{10}\ttfamily]
#define STM32F2_API_STM32_F20X_F21X_PORT_H_
	
#include "stm32_f20x_f21x_conf.h"\end{lstlisting}
	
	\item В случае, если файл стоит включать в процесс компиляции только при каком-то условии, это условие (обернутое в \textit{ifdef}) необходимо указать через одну пустую строку после \textit{include} файла конфигурации библиотеки. Блок \textit{endif}, закрывающий тело блока условной компиляции должен быть написан без пустой строки перед \textit{endif}, закрывающим блок защиты повторной компиляции.\\\textbf{Пример:}\begin{lstlisting}[language=C++, frame=tlBR, basicstyle=\fontsize{10}{10}\ttfamily]
#ifndef STM32F2_API_STM32_F20X_F21X_PORT_H_
#define STM32F2_API_STM32_F20X_F21X_PORT_H_
	
#include "stm32_f20x_f21x_conf.h"
	
#ifdef MODULE_PORT
	
CODE
	
#endif
#endif
 \end{lstlisting}
	\item Внутри всех необходимых обёрток, описанных выше (на месте слова <<CODE>> примера из предыдущего пункта), располагается основное содержимое, уникальное для каждого типа .h файла:\begin{itemize}
		\item \textit{perfix\_moduleName.h} (дел~\ref{p:modul:h});
		\item \textit{perfix\_moduleName\_constexpr\_func.h} (подраздел~\ref{p:conf:h});
		\item \textit{perfix\-\_moduleName\-\_constexpr\-\_func\-\_class\-\_className.h} (подраздел~\ref{p:conf:ch});
		\item \textit{perfix\_moduleName\_struct.h} и \textit{perfix\-\_moduleName\-\_struct\-\_class\-\_class\-Name.h}\\(подраздел~\ref{p:struc:h});
	\end{itemize}
\end{enumerate}

\subsection{Содержимое perfix\_moduleName.h файла}\label{p:modul:h}
\begin{enumerate}
	\item include \textit{perfix\_moduleName\_struct.h} и \textit{perfix\_moduleName\_constexpr\_func.h}  файлов, если таковые имеются;
	\item пустая строка;
	\item краткое описание всего модуля, включающее в себя описание всех классов модуля, обернутое в многострочный комментарий с явно обозначенными границами символами <<*>> в количестве 70 штук.
	\item пустая строка;
	\item краткое описание класса, обернутое в много строчный комментарий;
	\item пустая строка;
	\item include \textit{perfix\_moduleName\_struct\_class\_className.h}, если имеется;
	\item пустая строка;
	\item тело класса;
	\item пустая строка;
	\item include файла \textit{perfix\-\_moduleName\-\_constexpr\-\_func\-\_class\-\_className.h}, если имеется;
	\item пустая строка;
	\item пункты 5-12 повторить для всех требуемых классов;
	\item пустая строка.
\end{enumerate}
\textbf{Пример всего файла:}\begin{lstlisting}[language=C++, frame=tlBR, basicstyle=\fontsize{10}{10}\ttfamily]
#ifndef STM32F2_API_STM32_F20X_F21X_PORT_H_
#define STM32F2_API_STM32_F20X_F21X_PORT_H_

#include "stm32_f20x_f21x_conf.h"

#ifdef MODULE_PORT

#include "stm32_f20x_f21x_port_struct.h"						
#include "stm32_f20x_f21x_port_constexpr_func.h"				

/**********************************************************************
 * Краткое описание модуля...
 **********************************************************************/

/*
 * Краткое описание класса pin...
 */

#include "stm32_f20x_f21x_port_struct_class_pin.h"

class pin {
public:
	...

private:
	...
};

#include "stm32_f20x_f21x_port_constexpr_func_class_pin.h"

/*
 * Краткое описание класса global_port...
 */

#include "stm32_f20x_f21x_port_struct_class_global_port.h"		

class global_port {
public:
	...
private:
	...
};

#include "stm32_f20x_f21x_port_constexpr_func_class_global_port.h"

#endif
#endif
 \end{lstlisting}

\subsection{Содержимое perfix\-\_module\-Name\-\_const\-expr\-\_func.h файла}\label{p:conf:h}
\begin{enumerate}
	\item include \textit{perfix\-\_moduleName\-\_struct.h} файла, если таковой имеются;
	\item пустая строка;
	\item краткое описание constexpr функции, обернутое в много строчный комментарий;
	\item тело функции;
	\item пустая строка;
	\item пункты 3-5 повторить для всех имеющихся методов;
\end{enumerate}
\textbf{Пример всего файла:}\begin{lstlisting}[language=C++, frame=tlBR, basicstyle=\fontsize{10}{10}\ttfamily]
#ifndef STM32F2_API_PORT_STM32_F20X_F21X_PORT_CONSTEXPR_FUNC_H_
#define STM32F2_API_PORT_STM32_F20X_F21X_PORT_CONSTEXPR_FUNC_H_

#include "stm32_f20x_f21x_conf.h"

#ifdef MODULE_PORT

#include "stm32_f20x_f21x_port_struct.h"

/*
 * Краткое описание constexpr функции p_base_port_address_get...
 */

constexpr uint32_t p_base_port_address_get( EC_PORT_NAME port_name ) {
	CODE;
}

/*
 * Краткое описание constexpr функции bb_p_port_look_key_get...
 */
constexpr uint32_t bb_p_port_look_key_get ( EC_PORT_NAME port_name ) {
	CODE;
}

#endif
#endif
 \end{lstlisting}

\subsection{Содержимое perfix\-\_moduleName\-\_constexpr\-\_func\-\_class\-\_class-\\\-Name.h файла}\label{p:conf:ch}
\begin{enumerate}
	\item include \textit{perfix\_moduleName\_struct.h} и \textit{perfix\_moduleName\_struct\_class\_className.h} файлов, если таковые имеются;
	\item пустая строка;
	\item комментарий о начале области с constexpr constructor-ом(-ами), обернутый в многострочный комментарий с явно обозначенными границами символами <<*>> в количестве 70 штук. После комментария должна следовать  пустая строка;
	\item тело constexpr constructor конструктора;
	\item пустая строка;
	\item пункты 3-5 повторяются для всех имеющихся конструкторов;
	\item комментарий о начале области с constexpr методами, , обернутый в много строчный комментарий с явно прописанными границами, бросающимися в газа;
	\item пустая строка;
	\item краткое описание constexpr функции, обернутое в много строчный комментарий;
	\item тело функции;
	\item пустая строка;
	\item пункты 9-11 повторить для всех имеющихся методов;
	
\end{enumerate}
\textbf{Пример всего файла:}\begin{lstlisting}[language=C++, frame=tlBR, basicstyle=\fontsize{10}{10}\ttfamily]
#ifndef STM32F2_API_PORT_STM32_F20X_F21X_PORT_CONSTEXPR_FUNC_CLASS_PIN_H_
#define STM32F2_API_PORT_STM32_F20X_F21X_PORT_CONSTEXPR_FUNC_CLASS_PIN_H_

#include "stm32_f20x_f21x_conf.h"

#ifdef MODULE_PORT

#include "stm32_f20x_f21x_port_struct.h"
#include "stm32_f20x_f21x_port_struct_class_pin.h"

/**********************************************************************
 * Область constructor конструкторов.
 **********************************************************************/
constexpr pin::pin ( const pin_config_t *pin_cfg_array ):
	cfg								(pin_cfg_array),
	count							(1),
	p_port						(p_base_port_address_get	(pin_cfg_array->port)),
	p_bb_key_looking	(bb_p_port_look_key_get	(pin_cfg_array->port)),
	p_odr							(this->p_odr_get(pin_cfg_array)),
	p_bb_odr_read			(this->odr_bit_read_bb_p_get(pin_cfg_array)),
	set_msk						(this->set_msk_get(pin_cfg_array)),
	reset_msk					(this->reset_msk_get(pin_cfg_array)),
	p_bb_idr_read			(this->bb_p_idr_read_get(pin_cfg_array)),
	p_bb_looking_bit	(this->bb_p_looking_bit_get(pin_cfg_array)) {};

constexpr pin::pin ( const pin_config_t *pin_cfg_array, const uint32_t pin_cout ):
	КОД ИНИЦИАЛИЗАЦИИ ПЕРЕМЕННЫХ КЛАССА {};

/**********************************************************************
 * Область constexpr функций.
 **********************************************************************/
 
/*
 * Краткое описание constexpr функции set_msk_get...
 */
constexpr uint32_t pin::set_msk_get ( const pin_config_t *const pin_cfg_array ) {
	CODE;
}

/*
 * Краткое описание constexpr функции p_base_port_address_get...
 */
constexpr uint32_t pin::p_base_port_address_get ( const pin_config_t *const 
																									pin_cfg_array ) {
	CODE;
}

#endif
#endif
 \end{lstlisting}

\subsection{Содержимое perfix\_moduleName\_struct.h и\\erfix\_moduleName\_struct\_class\_className.h файлов}\label{p:struc:h}
В файлах располагаются следующие конструкции (сверху вниз):
\begin{enumerate}
	\item packed структуры;
	\item структуры;
	\item enum class-ы;
	\item макросы.
\end{enumerate}
Правила оформления следующие:
\begin{enumerate}
	\item перед каждым пунктом размещается комментарий о начале области этого пункта, обернутое в многострочный комментарий с явно обозначенными границами символами <<*>> в количестве 70 штук;
	\item перед каждым элементом пункта размещается краткое описание элемента, обернутое в много строчный комментарий. После краткого описания пустая строка не ставится;
	\item между предыдущим элементом пункта и комментарием следующего элемента этого же пункта вставляется пустая строка.
	\item между предыдущим элементом пункта и комментария области следующего пункта вставляется пустая строка.
\end{enumerate}
\textbf{Пример всего файла:}\begin{lstlisting}[language=C++, frame=tlBR, basicstyle=\fontsize{10}{10}\ttfamily]
#ifndef STM32F2_API_PORT_STM32_F20X_F21X_PORT_STRUCT_CLASS_PIN_H_
#define STM32F2_API_PORT_STM32_F20X_F21X_PORT_STRUCT_CLASS_PIN_H_

#include "stm32_f20x_f21x_conf.h"

#ifdef MODULE_PORT

/**********************************************************************
 * Область упакованных структур.
 **********************************************************************/
 
/*
 * Краткое описание упакованной структуры.
 */
struct __attribute__((packed)) port_registers_struct {
	ПОЛЯ СТРУКТУРЫ;
};

/**********************************************************************
 * Область структур.
 **********************************************************************/

/*
 * Краткое описание структуры.
 */
struct st_struct {
	ПОЛЯ СТРУКТУРЫ;
};

/**********************************************************************
 * Область enum class.
 **********************************************************************/
 
/*
 * Краткое описание num class-а.
 */
enum class EC_PORT_PIN_NAME {
	VALUE_FIELDS;
};


/**********************************************************************
 * Область макросов.
 **********************************************************************/

/*
 * Краткое описание макроса.
 */
#define M_PIN_CFG_ADC(PORT,PIN)	{
	VALUE_FIELDS;
}

#endif
#endif
 \end{lstlisting}