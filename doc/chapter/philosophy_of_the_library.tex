\chapter{ФИЛОСОФИЯ БИБЛИОТЕКИ}
\subsection{Общие сведения}
В основу библиотеки легли следующие постулаты:
\begin{enumerate}
	\item Все, что можно вычислить на этапе компиляции - не должно вычисляться в реальном времени. 
	\item Между производительностью и расходом памяти выбор должен быть в сторону производительности.
	\item Все, что может быть выполнено с помощью аппаратной периферии - не должно выполняться программно.
	\item Библиотека должна иметь как можно больше средств гибкой настройки на этапе компиляции и по минимуму - в реальном времени (в угоду производительности).
	\item Работа программы должна быть по максимуму предсказуема еще на этапе компиляции. Отсюда следует, что все режимы работы периферии должны быть заданы статически.
\end{enumerate}

\subsection{Краткий обзор реализации}
\begin{enumerate}
	\item Библиотека написана на C++14. 
	\item Большую часть библиотеки составляют constexpr функции, которые обрабатывают заполненные пользователем структуры инициализации периферии на этапе компиляции и создают маски регистров для всевозможных, указанных в структуре инициализации, режимов. В реальном времени созданные из const constexpr структур инициализации глобальные объекты в коде пользователя оперируют созданными на этапе компиляции масками регистров для настройки и работы с периферийными блоками.
	
	Этим достигается высокая производительность. Поскольку программе не нужно <<собирать>> маски регистров в реальном времени, как это сделано в HAL или SPL. Достаточно только применить маску.
	\item Тот факт, что для инициализации глобальных объектов используются глобальные const constexpr структуры вовсе не означает, что данные структуры войдут в состав прошивки контроллера.
	
	Яркий тому пример, объект класса global\_\-port (который будет рассмотрен в разделе~\ref{gp:0}). Он принимает в себя массив const constexpr pin\_config\_t структур, после чего private constexpr методы объекта класса global\_\-port их (структуры) анализируют и возвращают private global\_\-port\_\-msk\_\-reg\_\-struct структуру, которая будет private структурой глобального объекта класса global\_\-port.
	
	Структуры pin\_\-config\_\-t, использовавшиеся для инициализации private global\_\-port\_\-msk\_\-reg\_\-struct, во flash загружены не будут, потому что в ходе работы программы обращений к ним не будет.
\end{enumerate}

\label{gp:0}