\section{Модуль работы с портами ввода-вывода общего назначения (PORT)}\label{port:0}
Данный модуль состоит из следующих классов:
\begin{itemize}
	\item \textbf{pin} (раздел~\ref{class:pin});
	\item \textbf{global\_port} (раздел~\ref{class:global:port});
\end{itemize}

\subsection{Класс pin}\label{class:pin}
\subsubsection{Общие сведения о классе}
\begin{itemize}
	\item объект данного класса позволяет работать с конкретным выводом порта;
	\item при инициализации объекта данного класса в него необходимо передать указатель на структуру pin\_config, представляющую из себя  настройку вывода в каком-либо режиме;
\end{itemize}
. В отличии от объекта global\_port, объект pin использует их в реальном времени.
* В связи с чем они пойдут в итоговый файл прошивки (что скажется на его объеме).
* Важно понимать, что все структуры конфигурации вывода должны относится к одному физическому выводу (изменяя
* лишь режим его работы).
* Предполагается, что начальная инициализация всех выводов производится при инициализации всех портов объектом
* global\_port. Что избавляет от надобности начальной инициализации выводов.
* В случае, если в процессе работы программы будет нужно сменить режим работы вывода (при условии, что он не был
* заблокирован объектом global\_port), это можно сделать по средствам метода reinit, в который требуется передать
* номер (счет с 0), структуры инициализации в массиве, указанной при создании объекта.
