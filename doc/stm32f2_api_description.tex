\documentclass[a4paper, 12pt]{report}		% Документ является статьей на несколько глав.
\usepackage[left=25mm, top=20mm, right=10mm, bottom=20mm, nohead, nofoot]{geometry}
\usepackage [warn]{mathtext}				% Чтобы можно было использовать русские буквы в формулах, 
											% но в случае использования предупреждать об этом
\usepackage{placeins}
\usepackage [T2A]{fontenc}		            % Выбор внутренней TEX−кодировки.
\usepackage [utf8]{inputenc}		        % Выбор кодовой страницы документа.
\usepackage [english, russian]{babel}		% Выбор языка документа.
\usepackage{amsmath}						% Математика.
\usepackage{svg}
\usepackage{graphicx}						% Картинки.
\usepackage{xcolor}
\usepackage{indentfirst}					% Красная строка в начале абзаца.

%Настраиваем гиппер-ссылки.
\usepackage[pdfpagelayout=OneColumn, 		% pdf отображается как сплошная полоса из A4.
			colorlinks=true,				% Не нужно рисовать рамку вокруг ссылок, 
											% но при этом идет выделение цветом.
			linkcolor=blue					% Используем черный цвет для обозначения 
											% гиппер ссылок в оглавлении.					
]{hyperref}									% Запускаем работу с гиппер ссылками.
\setcounter{chapter}{0}						% Счет идет с 1, а не с 0 в оглавлении.

\begin{document}
	%---------------------------------------------------------------------------
	% Титульный лист.
	%---------------------------------------------------------------------------
	\title {ОПИСАНИЕ БИБЛИОТЕКИ STM32F2\_API}
	\author {Автор: Дерябкин Вадим (Vadimatorik)}
	\date {2017}
	\maketitle
	
	%---------------------------------------------------------------------------
	% Введение.
	%---------------------------------------------------------------------------
	\chapter{ВВЕДЕНИЕ}
	В данном документе приводится исчерпывающее описание:
	\begin{itemize}
		\item философии библиотеки (логики построения и использования)
		\item соглашения о написании библиотеки (допустимые синтаксические приемы языка и общие правила написания кода)
		\item примеров использования библиотеки в реальных задачах
	\end{itemize}

	\tableofcontents
	\clearpage							% Первая глава должна идти начиная со следущей страницы.
	
	%\input{chapter/RmMABA.tex}
	%\input{chapter/RmPWR.tex}
	%\input{chapter/RmGPIO.tex}
	%\input{chapter/RmRNG.tex}
\end{document}