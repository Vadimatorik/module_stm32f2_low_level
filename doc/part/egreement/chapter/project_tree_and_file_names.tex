\chapter{Дерево проекта и именование файлов}\label{dn:0}
Правила, касающиеся оформления библиотеки:
\begin{enumerate}
	\item Для файлов, относящихся к работе с блоками аппаратной и программной (абстрактные) периферии, должна существовать своя папка на каждый модуль.\\
	Пример: \textit{rcc}, \textit{port}, \textit{pwr} и т.д.\\
	Имя папки должно содержать только название аппаратного модуля, написанного строчными буквами латинского алфавита.
	
	\item Каждая папка, посвящённая определённому блоку периферии (аппаратной или программной), должна содержать следующие файлы:
	\begin{itemize}
		\item \textbf{perfix\-\_module\-Name.h}\\
		В данном файле должны находится классы, относящиеся к определённому блоку периферии. Объекты этих классов можно использовать в коде пользователя.
		\item \textbf{perfix\-\_module\-Name.cpp}\\
		Если в \textit{perfix\_moduleName.h} всего один класс, то в данном файле находятся методы класса из файла \textit{perfix\_moduleName.h}, вызов которых производится в реальном времени.\\
		В случае, если классов несколько и у них нет static общих методов (используемые двумя и более классами) - данный файл создавать не следует. Вместо этого для уникальных методов каждого класса должен быть свой файл с соответствующим постфиксом (именем класса). Об этом ниже.
		\item \textbf{perfix\-\_moduleName\-\_class\-\_class\-Name.cpp}\\
		В случае, если в файле \textit{perfix\-\_moduleName.h} более одного класса и какой-то из этих классов имеет методы, доступные только ему - их следует вынести в отдельный файл с постфиксом, соответствующим имени класса, к которому он (метод) относится.\\
		В случае, если в файле \textit{perfix\-\_module\-Name.h} один класс, методы, относящиеся к этому классу, должны быть размещены в файле \textit{perfix\_\-moduleName.cpp}.
		\item \textbf{perfix\-\_module\-Name\-\_constexpr\-\_func.h}\\
		В данном файле содержатся все constexpr методы, которые используются классом(-ами) из файла \textit{perfix\_moduleName.h}. Эти методы являются private методами класса(-ов).\\
		В случае, если в файле \textit{perfix\-\_module\-Name.h} более одного класса, в данном файле должны находятся лишь те методы, которые используются всеми классами файла \textit{perfix\-\_module\-Name.h}.\\
		В случае, если каждый класс файла \textit{perfix\_moduleName.h} использует лишь свой определённый набор методов, никак не пересекающийся с остальными классами, данный файл создавать не следует.
		\item \textbf{perfix\-\_moduleName\-\_constexpr\-\_func\-\_class\-\_class\-Name.h}\\
		В случае, если в файле \textit{perfix\_moduleName.h} более одного класса и у какого-то из классов имеются constexpr методы, никак не связанные с остальными (используются только им), их следует вынести в отдельный файл.\\
		В случае, если таких классов несколько (каждый из которых использует свои определённые constexpr методы), то для каждого такого класса следует создать отдельный файл.
		\item \textbf{perfix\_moduleName\_struct.h}\\		
		В данном файле содержатся все структуры и enum class-ы, используемые всеми классами файла \textit{perfix\_moduleName.h}.\\		
		В случае, если классы не имеют общих структур или enum class-ов, данный файл создавать не следует.\\		
		В случае, если в \textit{perfix\_moduleName.h} всего один класс, его структуры и enum class-ы должны располагаться здесь без создания конкретного файла под конкретный класс (из пункта ниже).
		\item \textbf{perfix\_moduleName\_struct\_class\_className.h}\\		
		В случае, если классов в файле \textit{perfix\_moduleName.h} более одного и у какого-то из классов имеются структуры или enum class-ы, которые используются только им одним, данные структуры и/или enum class-ы требуется вынести в отдельный файл с постфиксом имени класса, к которому они относятся.
	\end{itemize}

	Имена всех файлов должны быть написаны строчными латинскими символами (маленькие английские буквы). В том числе и сокращения по типу <<pwr>>.
	
	Все слова в имени должны разделяться символами нижнего подчеркивания. 
	
	В качестве примера рассмотрим дерево папки port библиотеки stm32\_f20x\_f21x (название библиотеки выступает в качестве префикса).
	
	stm32\_f20x\_f21x\_port.h содержит 2 класса (global\_port и pin). У них есть общие структуры, enum class-ы и методы. Однако есть и личные (используемые только ими) структуры, enum class-ы и constexpr методы. При этом у них нет общих static методов.
	\begin{lstlisting}[language=C++, frame=tlBR, basicstyle=\fontsize{10}{10}\ttfamily]
stm32_f20x_f21x_port_class_global_port.cpp
stm32_f20x_f21x_port_class_pin.cpp 
stm32_f20x_f21x_port_constexpr_func_class_global_port.h
stm32_f20x_f21x_port_constexpr_func_class_pin.h
stm32_f20x_f21x_port_constexpr_func.h
stm32_f20x_f21x_port_struct_class_global_port.h
stm32_f20x_f21x_port_struct_class_pin.h
stm32_f20x_f21x_port_struct.h
stm32_f20x_f21x_port.h\end{lstlisting}\end{enumerate}