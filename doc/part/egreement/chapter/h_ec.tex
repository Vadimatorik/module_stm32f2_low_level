\chapter{Объявление enum class-ов}\label{ec:h:0}
\section{Когда стоит объявлять enum class?}
Enum class-ы стоит объявлять, если какому-то полю структуры/переменной требуется присвоить какое-то одно значение из заранее известного ряда, каждому из которых можно присвоить уникальное имя.

\section{Размещение enum class-ов}
Прототипы enum class-ов должны быть размещены только в \textbf{.h} файлах.

\section{Оформление enum class-ов}
\begin{itemize}
	\item перед первым объявленным enum class-ом размещается комментарий о начале соответствующей области (области enum class-ов), обернутый в многострочный комментарий с явно обозначенными границами символами <<*>> в количестве 70 штук. После комментария должна следовать  пустая строка;
	\item перед каждым enum class-ом размещается его краткое описание, обернутое в многострочный комментарий. После краткого описания пустая строка не ставится;
	\item заголовок enum class-а следует оформить следующим образом:
	\begin{enumerate}
		\item ключевое словосочетание <<enum class>> без отступов в начале строки;
		\item отступ в один (1) пробел;
		\item имя enum class-а (в соответствии с правилами написания имен enum class-ов);
		\item пробел;
		\item открывающая тело enum class-а скобка <<\{>>;
	\end{enumerate}
	\item значения enum class-ов следует оформлять следующим образом:
	\begin{enumerate}
		\item каждое значение начинается с отступа в один (1) tab;
		\item имя значения;
		\item требуемое количество отступов, выполненных с помощью tab;
		\item <<=	>> (<<=>> + один (1) tab);
		\item непосредственное цифровое значение;
		\item <<,>> (в случае, если элемент последний, запятая не ставится).
		\item требуемое количество tab;
		\item <<// >> (// + пробел) + однострочный комментарий.
	\end{enumerate}
	\item выравнивание производится по знаку <<=>> с левой стороны;
	\item после последнего значения enum class-а следует скобка закрытия его тела и точка с запятой (<<\};>>), расположенная на новой строке;
	\item после последнего enum class - а вставляется пуста строка.
\end{itemize}\textbf{Пример области структур:}\begin{lstlisting}[language=C++, frame=tlBR, basicstyle=\fontsize{10}{10}\ttfamily]
/**********************************************************************
 * Область enum class- ов.
 **********************************************************************/

enum class EC_PORT_NAME {
	A	=	0,
	B	=	1,
	C	=	2,
	D	=	3,
	H	=	4
};\end{lstlisting}