\chapter{Правила оформления имён}\label{general:rules:0}
\begin{enumerate}
	\item Все имена переменных, структур, объектов, функций должны быть написаны строчными латинскими символами (маленькие английские буквы).
	
	\textbf{Пример: }<<pwr>>, <<port>>, <<value>>.
	\item Директивы препроцессора (define, макросы, ifndef и т.д.) должны писаться заглавными латинскими символами (большие английские буквы). 
	
	\textbf{Пример: }<<ADD(A,B)>>
	\item Слова в именах должны быть разделены нижним подчеркиванием.
	
	\textbf{Пример: }<<buf\_speed>>, <<STM32F2\-\_API\-\_PORT\-\_STM32\-\_F20X\-\_F21X\-\_PORT\-\_STRUCT\-\_\\CLASS\_PIN\-\_H\_>>, <<PORT\_PIN\_0>>
	\item Макросы должны начинаться с префикса <<M\_>>, после чего идет действие, которое он совершает (<<GET>>/<<SET>>).
	
	В именах так же следует использовать принятые сокращения.
	
	\textbf{Пример: }<<M\_GET\_BB\_P\_PER(ADDRESS,BIT)>>
	\item \textbf{Рекомендуется воздержаться от использования enum-ов}.
	
	Заместо них следует использовать \textbf{enum class}.
	\item Имя прототипа enum class должно начинаться с префикса <<EC\_>>. К нему можно обращаться только через <<::>>.
	
	Прямое обращение к значению enum class-а без указания пространства имен - запрещено.
	
	\textbf{Пример: }<<EC\_PORT\_NAME::A>>
\end{enumerate}
