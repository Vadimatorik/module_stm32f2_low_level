\documentclass[a4paper, 12pt]{report}		% Документ является статьей на несколько глав.
\usepackage[left=15mm, top=10mm, right=5mm, bottom=15mm, nohead, nofoot]{geometry}
\usepackage [warn]{mathtext}				% Чтобы можно было использовать русские буквы в формулах, 
											% но в случае использования предупреждать об этом
\usepackage{placeins}
\usepackage [T2A]{fontenc}		            % Выбор внутренней TEX−кодировки.
\usepackage [utf8]{inputenc}		        % Выбор кодовой страницы документа.
\usepackage [english, russian]{babel}		% Выбор языка документа.
\usepackage {amsmath}						% Математика.
\usepackage {svg}
\usepackage {graphicx}						% Картинки.
\usepackage {xcolor}
\usepackage {indentfirst}					% Красная строка в начале абзаца.
\usepackage {listings}
\usepackage	{import}

\lstdefinestyle{C++}{language=C++, style=numbers}
\usepackage{ucs}							% Для поддержки русских символов в lstlisting.
\lstset {
	language		= C++,					% Выбираем язык по умолчанию.
	extendedchars	= \true,				% Включаем не латиницу.
	keepspaces		= true,					% Чтобы не съедались пробелы.
	escapechar		= |,					% |«выпадаем» в LATEX|.
	frame			= tb,					% Рамка сверху и снизу.
	commentstyle	= \itshape,				% Шрифт для комментариев.
	stringstyle		= \bfseries,			% Шрифт для строк.
	tabsize			= 2						% Размер tab.
}


%Настраиваем гиппер-ссылки.
\usepackage[pdfpagelayout=OneColumn, 		% pdf отображается как сплошная полоса из A4.
			colorlinks=true,				% Не нужно рисовать рамку вокруг ссылок, 
											% но при этом идет выделение цветом.
			linkcolor=blue					% Используем черный цвет для обозначения 
											% гиппер ссылок в оглавлении.					
]{hyperref}									% Запускаем работу с гиппер ссылками.
\setcounter{chapter}{1}						% Счет идет с 1, а не с 0 в оглавлении.

\begin{document}
	\title {ВВЕДЕНИЕ В БИБЛИОТЕКУ STM32F2\_API}			% Титульный лист.
	\author {Автор: Дерябкин Вадим (Vadimatorik)}
	\date {2017}
	\maketitle

\subsubsection{ВВЕДЕНИЕ}
Данный документ создан с целью донести до пользователя:
\begin{itemize}
	\item основные сведения о библиотеке (часть~\ref{logic:bibl});
	\item соглашения о написании и оформлении библиотеки (часть~\ref{logic:bib2});
	\item структуру абстрактных аппаратных блоков периферии и примеры их использования (часть~\ref{module:op});
\end{itemize}

\tableofcontents						% Оглавление.
\clearpage								% Первая глава должна идти начиная со следущей страницы.

\subimport* {chapter/philosophy/}	{main}		% Знакомство с библиотекой.
\subimport* {chapter/egreement/}	{main}		% Соглашение о написании и оформлении библиотеки.
\subimport* {chapter/module/}		{main}

\end{document}