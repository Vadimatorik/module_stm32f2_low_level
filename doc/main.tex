\documentclass[a4paper, 12pt]{report}		% Документ является статьей на несколько глав.
\usepackage[left=15mm, top=10mm, right=5mm, bottom=15mm, nohead, nofoot]{geometry}
\usepackage [warn]{mathtext}				% Чтобы можно было использовать русские буквы в формулах, 
											% но в случае использования предупреждать об этом
\usepackage{placeins}
\usepackage [T2A]{fontenc}		            % Выбор внутренней TEX−кодировки.
\usepackage [utf8]{inputenc}		        % Выбор кодовой страницы документа.
\usepackage [english, russian]{babel}		% Выбор языка документа.
\usepackage {amsmath}						% Математика.
\usepackage {svg}
\usepackage {graphicx}						% Картинки.
\usepackage {xcolor}
\usepackage {indentfirst}					% Красная строка в начале абзаца.
\usepackage {listings}
\lstdefinestyle{C++}{language=C++, style=numbers}
\usepackage{ucs}							% Для поддержки русских символов в lstlisting.
\lstset {
	language		= C++,					% Выбираем язык по умолчанию.
	extendedchars	= \true,				% Включаем не латиницу.
	keepspaces		= true,					% Чтобы не съедались пробелы.
	escapechar		= |,					% |«выпадаем» в LATEX|.
	frame			= tb,					% Рамка сверху и снизу.
	commentstyle	= \itshape,				% Шрифт для комментариев.
	stringstyle		= \bfseries,			% Шрифт для строк.
	tabsize			= 2						% Размер tab.
}


%Настраиваем гиппер-ссылки.
\usepackage[pdfpagelayout=OneColumn, 		% pdf отображается как сплошная полоса из A4.
			colorlinks=true,				% Не нужно рисовать рамку вокруг ссылок, 
											% но при этом идет выделение цветом.
			linkcolor=blue					% Используем черный цвет для обозначения 
											% гиппер ссылок в оглавлении.					
]{hyperref}									% Запускаем работу с гиппер ссылками.
\setcounter{chapter}{0}						% Счет идет с 1, а не с 0 в оглавлении.

\begin{document}
	\title {ВВЕДЕНИЕ В БИБЛИОТЕКУ STM32F2\_API}			% Титульный лист.
	\author {Автор: Дерябкин Вадим (Vadimatorik)}
	\date {2017}
	\maketitle

\subsubsection{ВВЕДЕНИЕ}
Данный документ создан с целью донести до пользователя:
\begin{itemize}
	\item основные сведения о библиотеке (глава~\ref{logic:bibl});
	\item соглашения о написании и оформлении библиотеки (глава~\ref{logic:bib2});
	\item структуру абстрактных аппаратных блоков периферии и примеры их использования (глава~\ref{module:op});
\end{itemize}

\tableofcontents
\clearpage								% Первая глава должна идти начиная со следущей страницы.
	
\part{СОГЛАШЕНИЕ О НАПИСАНИИ И ОФОРМЛЕНИИ БИБЛИОТЕКИ}\label{logic:bib2}
\subimport	{./}	{introduction.tex}								% Введение.
\chapter{Средства сборки}\label{compgcc:0}
В основе библиотеки лежат constexpr функции, полноценная поддержка которых появилась в C++14. Отсюда следует вывод, что минимально возможная версия используемого языка - C++14. В случае, если в более поздних версиях будет несовместимость с C++14, следует внести изменения в библиотеку, решающие вопросы несовместимости по средствам проверки версии используемого стандарта языка и выбора совместимого с ним участка кода.

Для компиляции библиотеки следует использовать arm-none-eabi-g++ не старее (GNU Tools for ARM Embedded Processors 6-2017-q1-update) 6.3.1 20170215 (release) [ARM/embedded-6-branch revision 245512].				% Средства сборки.
\section{Дерево проекта и именование файлов}\label{dn:0}
Правила, касающиеся оформления библиотеки:
\begin{enumerate}
	\item Для файлов, относящихся к работе с блоками аппаратной и программной (абстрактные) периферии, должна существовать своя папка на каждый модуль.\\
	Пример: \textit{rcc}, \textit{port}, \textit{pwr} и т.д.\\
	Имя папки должно содержать только название аппаратного модуля, написанного строчными буквами латинского алфавита.
	
	\item Каждая папка, посвящённая определённому блоку периферии (аппаратной или программной), должна содержать следующие файлы:
	\begin{itemize}
		\item \textbf{perfix\_moduleName.h}\\
		В данном файле должны находится классы, относящиеся к определённому блоку периферии. Объекты этих классов можно использовать в коде пользователя.
		\item \textbf{perfix\_moduleName.cpp}\\
		Если в perfix\_moduleName.h всего один класс, то в данном файле находятся методы класса из файла perfix\_moduleName.h, вызов которых производится в реальном времени.\\
		В случае, если классов несколько и у них нет static общих методов (используемые двумя и более классами) - данный файл создавать не следует. Вместо этого для уникальных методов каждого класса должен быть свой файл с соответствующим постфиксом (именем класса). Об этом ниже.
		\item \textbf{perfix\_moduleName\_class\_className.cpp}\\
		В случае, если в файле perfix\_moduleName.h более одного класса и какой-то из этих классов имеет методы, доступные только ему - их следует вынести в отдельный файл с постфиксом, соответствующим имени класса, к которому он (метод) относится.\\
		В случае, если в файле perfix\_moduleName.h один класс, методы, относящиеся к этому классу, должны быть размещены в файле perfix\_\-moduleName.cpp.
		\item \textbf{perfix\_moduleName\_constexpr\_func.h}\\
		В данном файле содержатся все constexpr методы, которые используются классом(-и) из файла perfix\_moduleName.h. Эти методы, как правило, являются private методами класса(-ов).\\
		В случае, если в файле perfix\_moduleName.h более одного класса, в данном файле должны находятся лишь те методы, которые используются всеми классами файла perfix\_moduleName.h.\\
		В случае, если каждый класс файла perfix\_moduleName.h использует лишь свой определенный набор методов, никак не пересекающийся с остальными классами, данный файл создавать не следует.
		\item \textbf{perfix\_moduleName\_constexpr\_func\_class\_className.h}\\
		В случае, если в файле perfix\_moduleName.h более одного класса и у какого-то из классов имеются constexpr методы, никак не связанные с остальными (используются только им), их следует вынести в отдельный файл.\\
		В случае, если таких классов несколько (каждый из которых использует свои определенные constexpr методы), то для каждого такого класса следует создать отдельный файл.
		\item \textbf{perfix\_moduleName\_struct.h}\\		
		В данном файле содержатся все структуры и enum class-ы, используемые всеми классами файла perfix\_moduleName.h.\\		
		В случае, если классы не имеют общих структур или enum class-ов, данный файл создавать не следует.\\		
		В случае, если в perfix\_moduleName.h всего один класс, его структуры и enum class-ы должны располагаться здесь без создания конкретного файла под конкретный класс (из пункта ниже).
		\item \textbf{perfix\_moduleName\_struct\_class\_className.h}\\		
		В случае, если классов в файле perfix\_moduleName.h более одного и у какого-то из классов имеются структуры или enum class-ы, которые используются только им одним, данные структуры и/или enum class-ы требуется вынести в отдельный файл с постфиксом имени класса, к которому они относятся.
	\end{itemize}

	Имена всех файлов должны быть написаны строчными латинскими символами (маленькие английские буквы). В том числе и сокращения по типу <<pwr>>.
	
	Все слова в имени должны разделяться символами нижнего подчеркивания. 
	
	В качестве примера рассмотрим дерево папки port библиотеки stm32\_f20x\_f21x (название библиотеки выступает в качестве префикса).
	
	stm32\_f20x\_f21x\_port.h содержит 2 класса (global\_port и pin). У них есть общие структуры, enum class-ы и методы. Однако есть и личные (используемые только ими) структуры, enum class-ы и constexpr методы. При этом у них нет общих static методов.
	\begin{lstlisting}[language=C++, frame=tlBR, basicstyle=\fontsize{10}{10}\ttfamily]
stm32_f20x_f21x_port_class_global_port.cpp
stm32_f20x_f21x_port_class_pin.cpp 
stm32_f20x_f21x_port_constexpr_func_class_global_port.h
stm32_f20x_f21x_port_constexpr_func_class_pin.h
stm32_f20x_f21x_port_constexpr_func.h
stm32_f20x_f21x_port_struct_class_global_port.h
stm32_f20x_f21x_port_struct_class_pin.h
stm32_f20x_f21x_port_struct.h
stm32_f20x_f21x_port.h\end{lstlisting}\end{enumerate}	% Дерево проекта и именование файлов.
\section{Принятые сокращения}\label{sk:0}
\begin{enumerate}
	\item Если uint32\_t переменная содержит внутри себя адрес в памяти (является указателем), то перед ее именем должен быть префикс <<p\_>>.
	
	\textbf{Пример:} <<p\_target\_port>>.
	\item <<bit\_banding\_>> == <<bb\_>>
	
	Только в тексте (не применимо к коду).
	\item <<point\_bit\_banding\_bit\_address>> == <<bb\_p\_>>
	
	Когда uint32\_t переменная содержит адрес бита в bit banding области (является указателем).
\end{enumerate}						% Принятые сокращения.
\section{Правила оформления имён}\label{general:rules:0}
\begin{enumerate}
	\item Все имена переменных, структур, объектов, функций должны быть написаны строчными латинскими символами (маленькие английские буквы).
	
	\textbf{Пример: }<<pwr>>, <<port>>, <<value>>.
	\item Директивы препроцессора (define, макросы, ifndef и т.д.) должны писаться заглавными латинскими символами (большие английские буквы). 
	
	\textbf{Пример: }<<ADD(A,B)>>
	\item Слова в именах должны быть разделены нижним подчеркиванием.
	
	\textbf{Пример: }<<buf\_speed>>, <<STM32F2\_\-API\_\-PORT\_\-STM32\_\-F20X\_\-F21X\_\-PORT\_\-STRUCT\_\-CLASS\_\-PIN\_\-H\_>>, <<PORT\_PIN\_0>>
	\item Макросы должны начинаться с префикса <<M\_>>, после чего идет действие, которое он совершает (<<GET>>/<<SET>>).
	
	В именах так же следует использовать принятые сокращения.
	
	\textbf{Пример: }<<M\_GET\_BB\_P\_PER(ADDRESS,BIT)>>
	\item \textbf{Рекомендуется воздержаться от использования enum-ов}.
	
	Заместо них следует использовать \textbf{enum class}.
	\item Имя прототипа enum class должно начинаться с префикса <<EC\_>>. К нему можно обращаться только через <<::>>.
	
	Прямое обращение к значению enum class-а без указания пространства имен - запрещено.
	
	\textbf{Пример: }<<EC\_PORT\_NAME::A>>
\end{enumerate}
							% Правила оформления имён.
\chapter{Оформление .h файлов библиотеки}\label{file:h}
\section{Общее оформление}
\begin{enumerate}
	\item Файл должен включать в себя защиту от повторного включения в процесс компиляции по типу \textit{ifndef-define-endif}, оканчивающуюся пустой строкой.\\\textbf{Пример:}\begin{lstlisting}[language=C++, frame=tlBR, basicstyle=\fontsize{10}{10}\ttfamily]
#ifndef STM32F2_API_STM32_F20X_F21X_PORT_H_
#define STM32F2_API_STM32_F20X_F21X_PORT_H_
	
#endif
 \end{lstlisting}
	
	\item После \textit{define} строки защиты следует пустая строка, за которой располагается \textit{include} на файл конфигурации библиотеки, имеющий имя \textbf{perfix\_conf.h} (название библиотеки выступает в качестве префикса).\\\textbf{Пример:}\begin{lstlisting}[language=C++, frame=tlBR, basicstyle=\fontsize{10}{10}\ttfamily]
#define STM32F2_API_STM32_F20X_F21X_PORT_H_
	
#include "stm32_f20x_f21x_conf.h"\end{lstlisting}
	
	\item В случае, если файл стоит включать в процесс компиляции только при каком-то условии, это условие (обернутое в \textit{ifdef}) необходимо указать через одну пустую строку после \textit{include} файла конфигурации библиотеки. Блок \textit{endif}, закрывающий тело блока условной компиляции должен быть написан без пустой строки перед \textit{endif}, закрывающим блок защиты повторной компиляции.\\\textbf{Пример:}\begin{lstlisting}[language=C++, frame=tlBR, basicstyle=\fontsize{10}{10}\ttfamily]
#ifndef STM32F2_API_STM32_F20X_F21X_PORT_H_
#define STM32F2_API_STM32_F20X_F21X_PORT_H_
	
#include "stm32_f20x_f21x_conf.h"
	
#ifdef MODULE_PORT
	
CODE
	
#endif
#endif
 \end{lstlisting}
	\item Внутри всех необходимых обёрток, описанных выше (на месте слова <<CODE>> примера из предыдущего пункта), располагается основное содержимое, уникальное для каждого типа .h файла:\begin{itemize}
		\item \textit{perfix\_moduleName.h} (дел~\ref{p:modul:h});
		\item \textit{perfix\_moduleName\_constexpr\_func.h} (подраздел~\ref{p:conf:h});
		\item \textit{perfix\-\_moduleName\-\_constexpr\-\_func\-\_class\-\_className.h} (подраздел~\ref{p:conf:ch});
		\item \textit{perfix\_moduleName\_struct.h} и \textit{perfix\-\_moduleName\-\_struct\-\_class\-\_class\-Name.h}\\(подраздел~\ref{p:struc:h});
	\end{itemize}
\end{enumerate}

\section{Содержимое perfix\_moduleName.h файла}\label{p:modul:h}
\begin{enumerate}
	\item include \textit{perfix\_moduleName\_struct.h} и \textit{perfix\_moduleName\_constexpr\_func.h}  файлов, если таковые имеются;
	\item пустая строка;
	\item краткое описание всего модуля, включающее в себя описание всех классов модуля, обернутое в многострочный комментарий с явно обозначенными границами символами <<*>> в количестве 70 штук;
	\item пустая строка;
	\item краткое описание класса, обернутое в много строчный комментарий;
	\item пустая строка;
	\item include \textit{perfix\_moduleName\_struct\_class\_className.h}, если имеется;
	\item пустая строка;
	\item тело класса;
	\item пустая строка;
	\item include файла \textit{perfix\-\_moduleName\-\_constexpr\-\_func\-\_class\-\_className.h}, если имеется;
	\item пустая строка;
	\item пункты 5-12 повторить для всех требуемых классов;
	\item пустая строка.
\end{enumerate}
\textbf{Пример всего файла:}\begin{lstlisting}[language=C++, frame=tlBR, basicstyle=\fontsize{10}{10}\ttfamily]
#ifndef STM32F2_API_STM32_F20X_F21X_PORT_H_
#define STM32F2_API_STM32_F20X_F21X_PORT_H_

#include "stm32_f20x_f21x_conf.h"

#ifdef MODULE_PORT

#include "stm32_f20x_f21x_port_struct.h"						
#include "stm32_f20x_f21x_port_constexpr_func.h"				

/**********************************************************************
 * Краткое описание модуля...
 **********************************************************************/

/*
 * Краткое описание класса pin...
 */

#include "stm32_f20x_f21x_port_struct_class_pin.h"

class pin {
public:
	...

private:
	...
};

#include "stm32_f20x_f21x_port_constexpr_func_class_pin.h"

/*
 * Краткое описание класса global_port...
 */

#include "stm32_f20x_f21x_port_struct_class_global_port.h"		

class global_port {
public:
	...
private:
	...
};

#include "stm32_f20x_f21x_port_constexpr_func_class_global_port.h"

#endif
#endif
 \end{lstlisting}

\section{Содержимое perfix\-\_module\-Name\-\_const\-expr\-\_func.h файла}\label{p:conf:h}
\begin{enumerate}
	\item include \textit{perfix\-\_moduleName\-\_struct.h} файла, если таковой имеются;
	\item пустая строка;
	\item краткое описание constexpr функции, обернутое в много строчный комментарий;
	\item тело функции;
	\item пустая строка;
	\item пункты 3-5 повторить для всех имеющихся методов;
\end{enumerate}
\textbf{Пример всего файла:}\begin{lstlisting}[language=C++, frame=tlBR, basicstyle=\fontsize{10}{10}\ttfamily]
#ifndef STM32F2_API_PORT_STM32_F20X_F21X_PORT_CONSTEXPR_FUNC_H_
#define STM32F2_API_PORT_STM32_F20X_F21X_PORT_CONSTEXPR_FUNC_H_

#include "stm32_f20x_f21x_conf.h"

#ifdef MODULE_PORT

#include "stm32_f20x_f21x_port_struct.h"

/*
 * Краткое описание constexpr функции p_base_port_address_get...
 */

constexpr uint32_t p_base_port_address_get( EC_PORT_NAME port_name ) {
	CODE;
}

/*
 * Краткое описание constexpr функции bb_p_port_look_key_get...
 */
constexpr uint32_t bb_p_port_look_key_get ( EC_PORT_NAME port_name ) {
	CODE;
}

#endif
#endif
 \end{lstlisting}

\section{Содержимое perfix\-\_moduleName\-\_constexpr\-\_func\-\_class\-\_class-\\\-Name.h файла}\label{p:conf:ch}
\begin{enumerate}
	\item include \textit{perfix\_moduleName\_struct.h} и \textit{perfix\_moduleName\_struct\_class\_className.h} файлов, если таковые имеются;
	\item пустая строка;
	\item комментарий о начале области с constexpr конструктором(-ами), обернутый в многострочный комментарий с явно обозначенными границами символами <<*>> в количестве 70 штук. После комментария должна следовать  пустая строка;
	\item тело constexpr конструктора;
	\item пустая строка;
	\item пункты 3-5 повторяются для всех имеющихся конструкторов;
	\item комментарий о начале области с constexpr методами, , обернутый в много строчный комментарий с явно прописанными границами, бросающимися в газа;
	\item пустая строка;
	\item краткое описание constexpr функции, обернутое в много строчный комментарий;
	\item тело функции;
	\item пустая строка;
	\item пункты 9-11 повторить для всех имеющихся методов;
	
\end{enumerate}
\textbf{Пример всего файла:}\begin{lstlisting}[language=C++, frame=tlBR, basicstyle=\fontsize{10}{10}\ttfamily]
#ifndef STM32F2_API_PORT_STM32_F20X_F21X_PORT_CONSTEXPR_FUNC_CLASS_PIN_H_
#define STM32F2_API_PORT_STM32_F20X_F21X_PORT_CONSTEXPR_FUNC_CLASS_PIN_H_

#include "stm32_f20x_f21x_conf.h"

#ifdef MODULE_PORT

#include "stm32_f20x_f21x_port_struct.h"
#include "stm32_f20x_f21x_port_struct_class_pin.h"

/**********************************************************************
 * Область constexpr конструкторов.
 **********************************************************************/
constexpr pin::pin ( const pin_config_t* pin_cfg_array ):
	КОД ИНИЦИАЛИЗАЦИИ ПЕРЕМЕННЫХ КЛАССА {};

/**********************************************************************
 * Область constexpr функций.
 **********************************************************************/
 
/*
 * Краткое описание constexpr функции set_msk_get...
 */
constexpr uint32_t pin::set_msk_get ( const pin_config_t* const pin_cfg_array ) {
	CODE;
}

/*
 * Краткое описание constexpr функции p_base_port_address_get...
 */
constexpr uint32_t pin::p_base_port_address_get ( const pin_config_t* const 
																									pin_cfg_array ) {
	CODE;
}

#endif
#endif
 \end{lstlisting}

\section{Содержимое perfix\_moduleName\_struct.h и\\erfix\_moduleName\_struct\_class\_className.h файлов}\label{p:struc:h}
В файлах располагаются следующие конструкции (сверху вниз):
\begin{enumerate}
	\item Enum class-ы.
	\item Макросы.
	\item Packed структуры.
	\item Структуры.
\end{enumerate}
Правила оформления следующие:
\begin{enumerate}
	\item Перед каждым из выше перечисленных пунктов размещается комментарий о начале области конкретного пункта. Он оборачивается в многострочный комментарий с явно обозначенными границами. Граница представляют из себя строку символов <<*>>. Верхняя и нижняя строка должна содержать 70 разграничивающих символов. После выделенного комментария следует пустая строка.
	\item Перед каждым элементом пункта размещается краткое описание элемента, обернутое в много строчный комментарий. После краткого описания пустая строка не ставится.
	\item Между предыдущим элементом пункта и комментарием следующего элемента этого же пункта вставляется пустая строка.
	\item Между предыдущим элементом пункта и комментария области следующего пункта вставляется пустая строка.
\end{enumerate}
\textbf{Пример всего файла:}\begin{lstlisting}[language=C++, frame=tlBR, basicstyle=\fontsize{10}{10}\ttfamily]
#ifndef STM32F2_API_PORT_STM32_F20X_F21X_PORT_STRUCT_CLASS_PIN_H_
#define STM32F2_API_PORT_STM32_F20X_F21X_PORT_STRUCT_CLASS_PIN_H_

#include "stm32_f20x_f21x_conf.h"

#ifdef MODULE_PORT

/**********************************************************************
 * Область упакованных структур.
 **********************************************************************/
 
/*
 * Краткое описание упакованной структуры.
 */
struct __attribute__((packed)) port_registers_struct {
	ПОЛЯ СТРУКТУРЫ;
};

/**********************************************************************
 * Область структур.
 **********************************************************************/

/*
 * Краткое описание структуры.
 */
struct st_struct {
	ПОЛЯ СТРУКТУРЫ;
};

/**********************************************************************
 * Область enum class-ов.
 **********************************************************************/
 
/*
 * Краткое описание num class-а.
 */
enum class EC_PORT_PIN_NAME {
	VALUE_FIELDS;
};


/**********************************************************************
 * Область макросов.
 **********************************************************************/

/*
 * Краткое описание макроса.
 */
#define M_PIN_CFG_ADC(PORT,PIN)	{
	VALUE_FIELDS;
}

#endif
#endif
 \end{lstlisting}						% Оформление .h файлов библиотеки.
\section{Объявление классов в .h файлах}\label{class:0}
\begin{enumerate}
	\item Общие сведения об оформлении class-ов в .h файлах (подраздел~\ref{OBK}).
	\item В public области должны располагаться (с соблюдением последовательности сверху вниз):\begin{itemize}
		\item конструктор(-ы) класса (подраздел~\ref{K:0:0});
		\item constexpr конструктор(-ы) класса (подраздел~\ref{K:0:1});
		\item constexpr методы класса (подраздел~\ref{constexpr:0});
		\item доступные пользователю нестатические методы класса, выполняющиеся в реальном времени и возвращающие значение стандартного типа или указатель на стандартный тип данных (подраздел~\ref{dp:n:s});
		\item доступные пользователю нестатические методы класса, выполняющиеся в реальном времени и возвращающие значение нестандартного типа или указатель на нестандартный тип данных (подраздел~\ref{dp:n:n};
		\item доступные пользователю статические (static) методы класса, выполняющиеся в реальном времени и возвращающие значение стандартного типа или указатель на стандартный тип (подраздел~\ref{dp:s:s});
		\item доступные пользователю статические (static) методы класса, выполняющиеся в реальном времени и возвращающие значение нестандартного типа или указатель на нестандартный тип (подраздел~\ref{dp:s:n});
		\item открытие переменные и константы класса, доступные пользователю напрямую (подраздел~\ref{dp:op}).
	\end{itemize}
	\item В private область должны располагаться (с соблюдением последовательности сверху вниз):\begin{itemize}
		\item внутренние constexpr методы класса, возвращающие значение стандартного типа или указатель на стандартный тип данных (подраздел~\ref{zp:constexpr:s});
		\item внутренние constexpr методы класса, возвращающие значение нестандартного типа или указатель на нестандартный тип данных (подраздел~\ref{zp:constexpr:n});
		\item закрытые нестатические методы класса, выполняющиеся в реальном времени и возвращающие значение стандартного типа или указатель на стандартный тип данных (подраздел~\ref{zp:n:s});
		\item закрытые нестатические методы класса, выполняющиеся в реальном времени и возвращающие значение нестандартного типа или указатель на нестандартный тип данных (подраздел~\ref{zp:n:n});
		\item закрытые статические (static) методы класса, выполняющиеся в реальном времени и возвращающие значение стандартного типа или указатель на стандартный тип (подраздел~\ref{zp:s:s});
		\item закрытые статические (static) методы класса, выполняющиеся в реальном времени и возвращающие значение нестандартного типа или указатель на нестандартный тип (подраздел~\ref{zp:s:n});
		\item закрытые константы класса стандартных типов (подраздел~\ref{zp:const:s}).
		\item закрытые константы класса нестандартных типов (подраздел~\ref{zp:const:n}).
		\item закрытые переменные класса стандартных типов (подраздел~\ref{zp:pp:s}).
		\item закрытые переменные класса нестандартных типов (подраздел~\ref{zp:pp:n}).
	\end{itemize}
\end{enumerate}

\subsection{Общие сведения об оформлении class-ов в .h файлах}\label{OBK}
\begin{itemize}
	\item Между зарезервированным словом class и именем класса ставится один (1) пробел.
	\item Между последним символом имени класса и открывающейся фигурной скобкой ставится один (1) пробел.
	\item Сначала идет public, а за ним private область.
	\item <<\}>> (скобка закрывающая тело класса) должна находится на новой строке.
\end{itemize}\begin{lstlisting}[language=C++,frame=tlBR]
class name_class {
public:
private:
};\end{lstlisting}

\subsection{Конструктор(-ы) класса}\label{K:0:0}
Использование не constexpr конструкторов классов запрещено. Это связано с неочевидной последовательностью вызова конструкторов глобальных объектов, которая может привести к неверной инициализации объекта (если явно не указывать последовательность вызовов с помощью специальных директив компоновщика). Например, сначала будет предпринята попытка инициализировать внешнюю переферию (за пределами микроконтроллера), не инициализировав интерфейс, по которому она подключена.

В случае если пользователь все же создаст объект, конструктор которого будет требовать выполнения кода функции конструктора во время инициализации, вызов его метода инициализации произведен не будет (объект останется не инициализированным).

\subsection{Constexpr конструктор(-ы) класса}\label{K:0:1}
\begin{itemize}
	\item В случае, если конструкторов несколько, они должны быть расположены от большего количества входных параметров к меньшему.
	\item Реализация самого конструктора не должна находится в теле класса. Ее (реализацию конструктора) следует вынести в отдельный файл.
	\item Перед словом constexpr должен быть выполнен отступ в 1 tab.	
	\item Между словом constexpr и именем конструктора(-ов) ставится один (1) пробел.
	\item После имени конструктора должен быть выполнен один (1) пробел. 
	\item Аргументы конструктора(-ов) в скобках должны быть разделены <<, >> (запятая + пробел).
	\item Внутри скобок перечисления аргументов конструктора должен быть отступ в 1 пробел с каждой стороны.\\\textbf{Пример: } <<( uint32\_t a, uint8\_t b )>>.
\end{itemize}
\textbf{Пример:}\begin{lstlisting}[language=C++, frame=tlBR, basicstyle=\fontsize{8}{8}\ttfamily]
	constexpr pin ( const pin_config_t *pin_cfg_array, const uint32_t pin_cout );
	constexpr pin ( const pin_config_t *pin_cfg_array );
\end{lstlisting}

\subsection{Constexpr методы класса}\label{constexpr:0}
Размещение constexpr методов в разделе public запрещено и не имеет смысла.

\subsection{Доступные пользователю нестатические методы класса, выполняющиеся в реальном времени и возвращающие значение стандартного типа или указатель на стандартный тип данных}\label{dp:n:s}
\begin{itemize}
	\item Перед типом возвращаемого значения должен быть выполнен отступ в один (1) tab.
	\item Имена методов должны быть выравнены с помощью tab с остальными методами этого типа. Выравнивание методов других типов производится по иной сетке.
	\item Аргументы методов в скобках должны быть разделены <<, >> (запятая + пробел).
	\item Внутри скобок перечисления аргументов метода должен быть отступ в один (1) пробел с каждой стороны.\\\textbf{Пример: } <<( uint32\_t a, uint8\_t b )>>.
	\item В случае, если метод не изменяет данные класса, после параметров в скобках следует поставить один (1) пробел, после чего слово <<const;>>. <<;>> закрывает заголовок функции.
\end{itemize}
\textbf{Пример:}\begin{lstlisting}[language=C++, frame=tlBR, basicstyle=\fontsize{8}{8}\ttfamily]
	void	set		( void ) const;
	void	reset		( void ) const;
	void	invert		( void ) const;
	int	read		( void ) const;
\end{lstlisting}

\subsection{Доступные пользователю нестатические методы класса, выполняющиеся в реальном времени и возвращающие значение нестандартного типа или указатель на нестандартный тип данных}\label{dp:n:n}
\begin{itemize}
	\item В качестве нестандартного типа может выступать enum class или структура.
	\item В качестве указателя на нестандартный тип может выступать указатель на enum class переменную или структуру.
	\item Перед возвращаемым типом метода должен быть отступ в один (1) tab.
	\item Имена методов должны быть выравнены с помощью tab с остальными методами этого типа. Выравнивание методов других типов производится по иной сетке.
	\item Аргументы методов в скобках должны быть разделены <<, >> (запятая + пробел).
	\item Внутри скобок перечисления аргументов метода должен быть отступ в 1 пробел с каждой стороны.\\\textbf{Пример: } <<( uint32\_t a, uint8\_t b )>>.
\end{itemize}
\textbf{Пример:}\begin{lstlisting}[language=C++, frame=tlBR, basicstyle=\fontsize{8}{8}\ttfamily]
	E_ANSWER_GP	met_g	( void );
\end{lstlisting}

\subsection{Доступные пользователю статические (static) методы класса, выполняющиеся в реальном времени и возвращающие значение стандартного типа или указатель на стандартный тип}\label{dp:s:s}	\begin{itemize}
	\item Перед зарезервированным словом <<static>> должен быть отступ в один (1) tab.
	\item Между <<static>> и типом возвращаемого значения должен быть выполнен отступ в один (1) пробел.
	\item Имена методов должны быть выравнены с помощью tab с остальными методами этого типа. Выравнивание методов других типов производится по иной сетке.
	\item Аргументы методов в скобках должны быть разделены <<, >> (запятая + пробел).
	\item Внутри скобок перечисления аргументов метода должен быть отступ в 1 пробел с каждой стороны.\\\textbf{Пример: } <<( uint32\_t a, uint8\_t b )>>.
	\item Так как предполагается, что данный метод будет работать с объектом(-ами) класса, в котором(-ых) описан его заголовок, то первым аргументом метода должен быть указатель на void, который внутри класса будет разыменован в указатель на объект класса, в котором был объявлен.\\Используется указатель на void, а не на объект класса с целью совместимости с FreeRTOS, написанной на C.
	\item Первый аргумент метода следует называть <<void *obj>>.
\end{itemize}
\textbf{Пример:}\begin{lstlisting}[language=C++, frame=tlBR, basicstyle=\fontsize{8}{8}\ttfamily]
	static void	task_1	( void *obj );
	static int	m_1	( void *obj );
\end{lstlisting}

\subsection{Доступные пользователю статические (static) методы класса, выполняющиеся в реальном времени и возвращающие значение нестандартного типа или указатель на нестандартный тип}\label{dp:s:n}
\begin{itemize}
	\item В качестве нестандартного типа может выступать enum class или структура.
	\item В качестве указателя на нестандартный тип может выступать указатель на enum class переменную или структуру.
	\item Перед зарезервированным словом <<static>> должен быть отступ в один (1) tab.
	\item Между <<static>> и типом возвращаемого значения должен быть выполнен отступ в один (1) пробел.
	\item Имена методов должны быть выравнены с помощью tab с остальными методами этого типа. Выравнивание методов других типов производится по иной сетке.
	\item Аргументы методов в скобках должны быть разделены <<, >> (запятая + пробел).
	\item Внутри скобок перечисления аргументов метода должен быть отступ в 1 пробел с каждой стороны.\\\textbf{Пример: } <<( uint32\_t a, uint8\_t b )>>.
	\item Так как предполагается, что данный метод будет работать с объектом(-ами) класса, в котором(-ых) описан его заголовок, то первым аргументом метода должен быть указатель на void, который внутри класса будет разыменован в указатель на объект класса, в котором был объявлен.\\Используется указатель на void, а не на объект класса с целью совместимости с FreeRTOS, написанной на C.
	\item Первый аргумент метода следует называть <<void *obj>>.
\end{itemize}
\textbf{Пример:}\begin{lstlisting}[language=C++, frame=tlBR, basicstyle=\fontsize{8}{8}\ttfamily]
	static E_ANSWER_GP	met_a	( void *obj );
\end{lstlisting}

\subsection{Открытие переменные и константы класса, доступные пользователю напрямую}\label{dp:op}
Размещение переменных (изменяемых или заданных как const) в public области запрещено. Даже в случае, если требуется просто читать/записывать одну переменную, следует сделать отдельный метод(-ы) для этого. Так как прямое чтение данных из класса является нарушением ООП.

\subsection{Внутренние constexpr методы класса, возвращающие значение стандартного типа или указатель на стандартный тип данных}\label{zp:constexpr:s}
\begin{itemize}
	\item Реализация тела функции не должна находится в теле класса. Она (реализация тела функции) должна быть вынесена в отдельный файл. Допускаются только заголовки функций.
	\item Перед словом constexpr должен быть выполнен отступ в один (1) tab.
	\item Между зарезервированным словом constexpr и типом возвращаемого значения требуется поставить один (1) пробел.
	\item Имена методов должны быть выравнены с помощью tab с остальными методами этого типа. Выравнивание методов других типов производится по иной сетке.
	\item Аргументы методов в скобках должны быть разделены <<, >> (запятая + пробел).
	\item Внутри скобок перечисления аргументов метода должен быть отступ в 1 пробел с каждой стороны.\\\textbf{Пример: } <<( uint32\_t a, uint8\_t b )>>.
\end{itemize}
\textbf{Пример:}\begin{lstlisting}[language=C++, frame=tlBR, basicstyle=\fontsize{8}{8}\ttfamily]
	constexpr uint32_t	moder_reg_reset_init_msk_get	( EC_PORT_NAME port_name );
\end{lstlisting}

\subsection{Внутренние constexpr методы класса, возвращающие значение нестандартного типа или указатель на нестандартный тип данных}\label{zp:constexpr:n}
\begin{itemize}
	\item В качестве нестандартного типа может выступать enum class или структура.
	\item В качестве указателя на нестандартный тип может выступать указатель на enum class переменную или структуру.
	\item Реализация тела функции не должна находится в теле класса. Она (реализация тела функции) должна быть вынесена в отдельный файл. Допускаются только заголовки функций.
	\item Перед словом constexpr должен быть выполнен отступ в один (1) tab.
	\item Между зарезервированным словом constexpr и типом возвращаемого значения требуется поставить один (1) пробел.
	\item Имена методов должны быть выравнены с помощью tab с остальными методами этого типа. Выравнивание методов других типов производится по иной сетке.
	\item Аргументы методов в скобках должны быть разделены <<, >> (запятая + пробел).
	\item Внутри скобок перечисления аргументов метода должен быть отступ в 1 пробел с каждой стороны.\\\textbf{Пример: } <<( uint32\_t a, uint8\_t b )>>.
\end{itemize}\begin{lstlisting}[language=C++, frame=tlBR, basicstyle=\fontsize{8}{8}\ttfamily]
	constexpr EC_PORT_NAME	port_name_get	( uint32_t value_reg );
\end{lstlisting}

\subsection{Закрытые нестатические методы класса, выполняющиеся в реальном времени и возвращающие значение стандартного типа или указатель на стандартный тип данных}\label{zp:n:s}
Оформляются так же, как и открытые нестатические методы класса, выполняющиеся в реальном времени и возвращающие значение стандартного типа или указатель на стандартный тип данных (подраздел~\ref{dp:n:s}).

\subsection{Закрытые нестатические методы класса, выполняющиеся в реальном времени и возвращающие значение нестандартного типа или указатель на нестандартный тип данных}\label{zp:n:n}
Оформляются так же, как и открытые нестатические методы класса, выполняющиеся в реальном времени и возвращающие значение нестандартного типа или указатель на нестандартный тип данных (подраздел~\ref{dp:n:n}).

\subsection{Закрытые статические (static) методы класса, выполняющиеся в реальном времени и возвращающие значение стандартного типа или указатель на стандартный тип}\label{zp:s:s}
Оформляются так же, как и открытые статические (static) методы класса, выполняющиеся в реальном времени и возвращающие значение стандартного типа или указатель на стандартный тип (подраздел~\ref{dp:s:s}).

\subsection{Закрытые статические (static) методы класса, выполняющиеся в реальном времени и возвращающие значение нестандартного типа или указатель на нестандартный тип}\label{zp:s:n}
Оформляются так же, как и открытые статические (static) методы класса, выполняющиеся в реальном времени и возвращающие значение нестандартного типа или указатель на нестандартный тип (подраздел~\ref{dp:s:n}).

\subsection{Закрытые константы класса стандартных типов}\label{zp:const:s}
\begin{itemize}
	\item На одной строке допустимо объявлять лишь одну константу.
	\item Перед зарезервированным словом <<const>> должен быть поставлен один (1) tab.
	\item Имена констант должны быть выравнены с помощью tab с остальными константами этого типа. Выравнивание констант других типов производится по иной сетке.
\end{itemize}\begin{lstlisting}[language=C++, frame=tlBR, basicstyle=\fontsize{8}{8}\ttfamily]
	const uint32_t	count;
\end{lstlisting}

\subsection{Закрытые константы класса нестандартных типов}\label{zp:const:n}
Оформляются так же, как и закрытые константы класса стандартных типов (подраздел~\ref{zp:const:s})\begin{lstlisting}[language=C++, frame=tlBR, basicstyle=\fontsize{8}{8}\ttfamily]
	const global_port_msk_reg_struct	gb_msk_struct;
\end{lstlisting}

\subsection{Закрытые переменные класса стандартных типов}\label{zp:pp:s}
\begin{itemize}
	\item На одной строке допустимо объявлять лишь одну переменную.
	\item Перед типом должен быть поставлен один (1) tab.
	\item Имена переменных должны быть выравнены с помощью tab с остальными переменными этого типа. Выравнивание переменных других типов производится по иной сетке.
\end{itemize}\begin{lstlisting}[language=C++, frame=tlBR, basicstyle=\fontsize{8}{8}\ttfamily]
	uint32_t	flag;
\end{lstlisting}

\subsection{Закрытые переменные класса нестандартных типов}\label{zp:pp:n}
Оформляются так же, как и закрытые переменные класса стандартных типов (подраздел~\ref{zp:pp:s})\begin{lstlisting}[language=C++, frame=tlBR, basicstyle=\fontsize{8}{8}\ttfamily]
	EC_FL	mb_msk_struct;
\end{lstlisting}						% Объявление классов в .h файлах.
\section{Объявление packed структур}\label{struct:p}
\subsection{Когда стоит объявлять структуру как packed?}
Структуру следует объявить как packed, если она:
\begin{itemize}
	\item описывает совокупность регистров аппаратного блока периферии (у которого, как известно, регистры имеют четко фиксированный размер и порядок следования);
	\item описывает образ памяти регистров аппаратного блока;
	\item описывает структуру какого-либо пакета со строго фиксированными полями.
\end{itemize}

\subsection{Размещение packed структур}
Packed структуры должны быть размещены только в \textbf{.h} файлах. 

\subsection{Оформление packed структур}
\begin{itemize}
	\item перед первой объявленной packed структурой размещается комментарий о начале области packed структур, обернутый в много строчный комментарий с явно обозначенными границами, бросающимися в газа. После чего вставляется пустая строка;
	\item перед каждой packed структурой размещается ее краткое описание, обернутое в много строчный комментарий. После краткого описания пустая строка не ставится;
	\item заголовок структуры следует оформить следующим образом:
	\begin{enumerate}
		\item ключевое слово struct без отступов в начале строки;
		\item отступ в один (1) пробел;
		\item директива препроцессора <<\_\_attribute\_\_((packed))>>;
		\item пробел;
		\item имя структуры;
		\item пробел;
		\item открывающая тело packed структуры скобка <<\{>>;
	\end{enumerate}
	\item поля структуры следует оформлять следующим образом:
	\begin{enumerate}
		\item каждая строка начинается с отступа в один (1) tab;
		\item ключевое слово volatile;
		\item один (1) пробел;
		\item тип поля;
		\item требуемое количество отступов, выполненных с помощью tab;
		\item имя поля;
		\item <<;>>;
		\item требуемое количество tab;
		\item <<// >> (// + пробел) + одно строчный комментарий.
	\end{enumerate}
	\item все имена полей структуры должны быть выравнены с помощью tab между собой;
	\item после последнего поля структуры следует скобка закрытия тела структуры (<<\}>>);
	\item после последней packed структуры вставляется пуста строка.
\end{itemize}\textbf{Пример packed области:}\begin{lstlisting}[language=C++, frame=tlBR, basicstyle=\fontsize{9}{9}\ttfamily]
/**************************************************************
 * Packed struct label.
 **************************************************************/

/*
 * Packed struct description...
 */
struct __attribute__((packed)) port_registers_struct {
	volatile uint32_t	mode;		// Comment...
	volatile uint32_t	otype;		// Comment...
	...;
};

\end{lstlisting}					% Объявление packed структур.
\chapter{Объявление структур}\label{struct}
\section{Когда стоит оборачивать данные в структуру?}
Данные следует обернуть в структуру, если:
\begin{itemize}
	\item требуется передавать более одного параметра в конструктор класса;
	\item требуется вернуть из функции более одного параметра;
\end{itemize}
\textbf{Замечание: }перед тем, как оборачивать данные в структуру проверьте, не имеются ли у вас условий, согласно которым данные должны быть обернуты в упакованную структуру (раздел~\ref{struct:p})
\section{Размещение структур}
Прототипы структур должны быть размещены только в \textbf{.h} файлах. Экземпляры - в \textbf{.cpp}.

\section{Оформление структур}
\begin{itemize}
	\item перед первой объявленной структурой размещается комментарий о начале соответствующей области (области структур), обернутый в многострочный комментарий с явно обозначенными границами символами <<*>> в количестве 70 штук. После комментария должна следовать  пустая строка;
	\item перед каждой структурой размещается ее краткое описание, обернутое в многострочный комментарий. После краткого описания пустая строка не ставится;
	\item заголовок структуры следует оформить следующим образом:
	\begin{enumerate}
		\item ключевое слово struct без отступов в начале строки;
		\item отступ в один (1) пробел;
		\item имя структуры;
		\item пробел;
		\item открывающая тело packed структуры скобка <<\{>>;
	\end{enumerate}
	\item поля структуры следует оформлять следующим образом:
	\begin{enumerate}
		\item каждая строка начинается с отступа в один (1) tab;
		\item тип поля;
		\item требуемое количество отступов, выполненных с помощью tab;
		\item имя поля;
		\item <<;>>;
		\item требуемое количество tab;
		\item <<// >> (// + пробел) + одно строчный комментарий.
	\end{enumerate}
	\item все имена полей структуры должны быть выравнены с помощью tab между собой;
	\item после последнего поля структуры следует скобка закрытия тела структуры (<<\}>>);
	\item после последней структуры вставляется пуста строка.
\end{itemize}\textbf{Пример области структур:}\begin{lstlisting}[language=C++, frame=tlBR, basicstyle=\fontsize{10}{10}\ttfamily]
/**********************************************************************
 * Область структур.
 **********************************************************************/

/*
 * Краткое описание структуры...
 */
struct a {
	uint32_t	b;		// Пояснение к полю b.
	uint32_t	c;		// Пояснение к полю c.
};\end{lstlisting}							% Объявление структур.
\chapter{Объявление enum class-ов}\label{ec:h:0}
\section{Когда стоит объявлять enum class?}
Enum class-ы стоит объявлять, если какому-то полю структуры/переменной требуется присвоить какое-то одно значение из заранее известного ряда, каждому из которых можно присвоить уникальное имя.

\section{Размещение enum class-ов}
Прототипы enum class-ов должны быть размещены только в \textbf{.h} файлах.

\section{Оформление enum class-ов}
\begin{itemize}
	\item перед первым объявленным enum class-ом размещается комментарий о начале соответствующей области (области enum class-ов), обернутый в многострочный комментарий с явно обозначенными границами символами <<*>> в количестве 70 штук. После комментария должна следовать  пустая строка;
	\item перед каждым enum class-ом размещается его краткое описание, обернутое в многострочный комментарий. После краткого описания пустая строка не ставится;
	\item заголовок enum class-а следует оформить следующим образом:
	\begin{enumerate}
		\item ключевое словосочетание <<enum class>> без отступов в начале строки;
		\item отступ в один (1) пробел;
		\item имя enum class-а (в соответствии с правилами написания имен enum class-ов);
		\item пробел;
		\item открывающая тело enum class-а скобка <<\{>>;
	\end{enumerate}
	\item значения enum class-ов следует оформлять следующим образом:
	\begin{enumerate}
		\item каждое значение начинается с отступа в один (1) tab;
		\item имя значения;
		\item требуемое количество отступов, выполненных с помощью tab;
		\item <<=	>> (<<=>> + один (1) tab);
		\item непосредственное цифровое значение;
		\item <<,>> (в случае, если элемент последний, запятая не ставится).
		\item требуемое количество tab;
		\item <<// >> (// + пробел) + однострочный комментарий.
	\end{enumerate}
	\item выравнивание производится по знаку <<=>> с левой стороны;
	\item после последнего значения enum class-а следует скобка закрытия его тела и точка с запятой (<<\};>>), расположенная на новой строке;
	\item после последнего enum class - а вставляется пуста строка.
\end{itemize}\textbf{Пример области структур:}\begin{lstlisting}[language=C++, frame=tlBR, basicstyle=\fontsize{10}{10}\ttfamily]
/**********************************************************************
 * Область enum class- ов.
 **********************************************************************/

enum class EC_PORT_NAME {
	A	=	0,
	B	=	1,
	C	=	2,
	D	=	3,
	H	=	4
};\end{lstlisting}							% Объявление enum class-ов.
\chapter{Объявление макросов}\label{mackros:0}
\section{Когда стоит объявлять макрос?}
Макрос следует объявить, если имеется фрагмент кода, многократно повторяющийся в коде программы с незначительными изменениями в каждом конкретном случае. При этом использовать отдельную constexpr функцию нерационально или невозможно.\\Пример: при заполнении массива структур инициализации выводов некоторое количество выводов инициализируются как входы ADC. Для того, чтобы не писать каждый раз все параметры каждого пина, достаточно будет воспользоваться макросом, в котором нужно указать изменяющиеся параметры: имя порта и вывода.

\section{Размещение макросов}
Прототипы макросов должны быть размещены в \textbf{.h} файлах.

\section{Оформление макросов}
\begin{itemize}
	\item перед первым объявленным макросом размещается комментарий о начале соответствующей области (области макросов), обернутый в многострочный комментарий с явно обозначенными границами символами <<*>> в количестве 70 штук. После комментария должна следовать  пустая строка;
	\item перед каждым макросом размещается его краткое описание, обернутое в многострочный комментарий. После краткого описания пустая строка не ставится;
	\item заголовок макроса следует оформить следующим образом:
	\begin{enumerate}
		\item ключевое слово \#define без отступов в начале строки;
		\item отступ в один (1) пробел;
		\item имя макроса (в соответствии с правилами написания макросов);
		\item открывающаяся скобка (<<(>>);
		\item параметры макроса через запятую без пробелов в соответствии с правилами написания макросов;
		\item закрывающаяся скобка (<<)>>);
		\item тело макроса;
	\end{enumerate}
\end{itemize}\textbf{Пример:}\begin{lstlisting}[language=C++, frame=tlBR, basicstyle=\fontsize{10}{10}\ttfamily]
/**********************************************************************
 * Область макросов.
 **********************************************************************/

/*
 * Возвращает структуру конфигурации вывода,
 * настроенного на вход, подключенный к ADC.
 */
#define M_PIN_CFG_ADC(PORT,PIN)	{											\
	.port							= PORT,														\
	.pin_name					= PIN,														\
	.mode							= EC_PIN_MODE::ANALOG,						\
	.output_config		= EC_PIN_OUTPUT_CFG::NOT_USE,			\
	.speed						= EC_PIN_SPEED::LOW,							\
	.pull							= EC_PIN_PULL::NO,								\
	.af								= EC_PIN_AF::NOT_USE,							\
	.locked						= EC_LOCKED::LOCKED,							\
	.state_after_init	= EC_PIN_STATE_AFTER_INIT::NO_USE	\
}\end{lstlisting}						% Объявление макросов.

\if 0
\fi		% Знакомство с библиотекой.
\part{СОГЛАШЕНИЕ О НАПИСАНИИ И ОФОРМЛЕНИИ БИБЛИОТЕКИ}\label{logic:bib2}
\subimport	{./}	{introduction.tex}								% Введение.
\chapter{Средства сборки}\label{compgcc:0}
В основе библиотеки лежат constexpr функции, полноценная поддержка которых появилась в C++14. Отсюда следует вывод, что минимально возможная версия используемого языка - C++14. В случае, если в более поздних версиях будет несовместимость с C++14, следует внести изменения в библиотеку, решающие вопросы несовместимости по средствам проверки версии используемого стандарта языка и выбора совместимого с ним участка кода.

Для компиляции библиотеки следует использовать arm-none-eabi-g++ не старее (GNU Tools for ARM Embedded Processors 6-2017-q1-update) 6.3.1 20170215 (release) [ARM/embedded-6-branch revision 245512].				% Средства сборки.
\section{Дерево проекта и именование файлов}\label{dn:0}
Правила, касающиеся оформления библиотеки:
\begin{enumerate}
	\item Для файлов, относящихся к работе с блоками аппаратной и программной (абстрактные) периферии, должна существовать своя папка на каждый модуль.\\
	Пример: \textit{rcc}, \textit{port}, \textit{pwr} и т.д.\\
	Имя папки должно содержать только название аппаратного модуля, написанного строчными буквами латинского алфавита.
	
	\item Каждая папка, посвящённая определённому блоку периферии (аппаратной или программной), должна содержать следующие файлы:
	\begin{itemize}
		\item \textbf{perfix\_moduleName.h}\\
		В данном файле должны находится классы, относящиеся к определённому блоку периферии. Объекты этих классов можно использовать в коде пользователя.
		\item \textbf{perfix\_moduleName.cpp}\\
		Если в perfix\_moduleName.h всего один класс, то в данном файле находятся методы класса из файла perfix\_moduleName.h, вызов которых производится в реальном времени.\\
		В случае, если классов несколько и у них нет static общих методов (используемые двумя и более классами) - данный файл создавать не следует. Вместо этого для уникальных методов каждого класса должен быть свой файл с соответствующим постфиксом (именем класса). Об этом ниже.
		\item \textbf{perfix\_moduleName\_class\_className.cpp}\\
		В случае, если в файле perfix\_moduleName.h более одного класса и какой-то из этих классов имеет методы, доступные только ему - их следует вынести в отдельный файл с постфиксом, соответствующим имени класса, к которому он (метод) относится.\\
		В случае, если в файле perfix\_moduleName.h один класс, методы, относящиеся к этому классу, должны быть размещены в файле perfix\_\-moduleName.cpp.
		\item \textbf{perfix\_moduleName\_constexpr\_func.h}\\
		В данном файле содержатся все constexpr методы, которые используются классом(-и) из файла perfix\_moduleName.h. Эти методы, как правило, являются private методами класса(-ов).\\
		В случае, если в файле perfix\_moduleName.h более одного класса, в данном файле должны находятся лишь те методы, которые используются всеми классами файла perfix\_moduleName.h.\\
		В случае, если каждый класс файла perfix\_moduleName.h использует лишь свой определенный набор методов, никак не пересекающийся с остальными классами, данный файл создавать не следует.
		\item \textbf{perfix\_moduleName\_constexpr\_func\_class\_className.h}\\
		В случае, если в файле perfix\_moduleName.h более одного класса и у какого-то из классов имеются constexpr методы, никак не связанные с остальными (используются только им), их следует вынести в отдельный файл.\\
		В случае, если таких классов несколько (каждый из которых использует свои определенные constexpr методы), то для каждого такого класса следует создать отдельный файл.
		\item \textbf{perfix\_moduleName\_struct.h}\\		
		В данном файле содержатся все структуры и enum class-ы, используемые всеми классами файла perfix\_moduleName.h.\\		
		В случае, если классы не имеют общих структур или enum class-ов, данный файл создавать не следует.\\		
		В случае, если в perfix\_moduleName.h всего один класс, его структуры и enum class-ы должны располагаться здесь без создания конкретного файла под конкретный класс (из пункта ниже).
		\item \textbf{perfix\_moduleName\_struct\_class\_className.h}\\		
		В случае, если классов в файле perfix\_moduleName.h более одного и у какого-то из классов имеются структуры или enum class-ы, которые используются только им одним, данные структуры и/или enum class-ы требуется вынести в отдельный файл с постфиксом имени класса, к которому они относятся.
	\end{itemize}

	Имена всех файлов должны быть написаны строчными латинскими символами (маленькие английские буквы). В том числе и сокращения по типу <<pwr>>.
	
	Все слова в имени должны разделяться символами нижнего подчеркивания. 
	
	В качестве примера рассмотрим дерево папки port библиотеки stm32\_f20x\_f21x (название библиотеки выступает в качестве префикса).
	
	stm32\_f20x\_f21x\_port.h содержит 2 класса (global\_port и pin). У них есть общие структуры, enum class-ы и методы. Однако есть и личные (используемые только ими) структуры, enum class-ы и constexpr методы. При этом у них нет общих static методов.
	\begin{lstlisting}[language=C++, frame=tlBR, basicstyle=\fontsize{10}{10}\ttfamily]
stm32_f20x_f21x_port_class_global_port.cpp
stm32_f20x_f21x_port_class_pin.cpp 
stm32_f20x_f21x_port_constexpr_func_class_global_port.h
stm32_f20x_f21x_port_constexpr_func_class_pin.h
stm32_f20x_f21x_port_constexpr_func.h
stm32_f20x_f21x_port_struct_class_global_port.h
stm32_f20x_f21x_port_struct_class_pin.h
stm32_f20x_f21x_port_struct.h
stm32_f20x_f21x_port.h\end{lstlisting}\end{enumerate}	% Дерево проекта и именование файлов.
\section{Принятые сокращения}\label{sk:0}
\begin{enumerate}
	\item Если uint32\_t переменная содержит внутри себя адрес в памяти (является указателем), то перед ее именем должен быть префикс <<p\_>>.
	
	\textbf{Пример:} <<p\_target\_port>>.
	\item <<bit\_banding\_>> == <<bb\_>>
	
	Только в тексте (не применимо к коду).
	\item <<point\_bit\_banding\_bit\_address>> == <<bb\_p\_>>
	
	Когда uint32\_t переменная содержит адрес бита в bit banding области (является указателем).
\end{enumerate}						% Принятые сокращения.
\section{Правила оформления имён}\label{general:rules:0}
\begin{enumerate}
	\item Все имена переменных, структур, объектов, функций должны быть написаны строчными латинскими символами (маленькие английские буквы).
	
	\textbf{Пример: }<<pwr>>, <<port>>, <<value>>.
	\item Директивы препроцессора (define, макросы, ifndef и т.д.) должны писаться заглавными латинскими символами (большие английские буквы). 
	
	\textbf{Пример: }<<ADD(A,B)>>
	\item Слова в именах должны быть разделены нижним подчеркиванием.
	
	\textbf{Пример: }<<buf\_speed>>, <<STM32F2\_\-API\_\-PORT\_\-STM32\_\-F20X\_\-F21X\_\-PORT\_\-STRUCT\_\-CLASS\_\-PIN\_\-H\_>>, <<PORT\_PIN\_0>>
	\item Макросы должны начинаться с префикса <<M\_>>, после чего идет действие, которое он совершает (<<GET>>/<<SET>>).
	
	В именах так же следует использовать принятые сокращения.
	
	\textbf{Пример: }<<M\_GET\_BB\_P\_PER(ADDRESS,BIT)>>
	\item \textbf{Рекомендуется воздержаться от использования enum-ов}.
	
	Заместо них следует использовать \textbf{enum class}.
	\item Имя прототипа enum class должно начинаться с префикса <<EC\_>>. К нему можно обращаться только через <<::>>.
	
	Прямое обращение к значению enum class-а без указания пространства имен - запрещено.
	
	\textbf{Пример: }<<EC\_PORT\_NAME::A>>
\end{enumerate}
							% Правила оформления имён.
\chapter{Оформление .h файлов библиотеки}\label{file:h}
\section{Общее оформление}
\begin{enumerate}
	\item Файл должен включать в себя защиту от повторного включения в процесс компиляции по типу \textit{ifndef-define-endif}, оканчивающуюся пустой строкой.\\\textbf{Пример:}\begin{lstlisting}[language=C++, frame=tlBR, basicstyle=\fontsize{10}{10}\ttfamily]
#ifndef STM32F2_API_STM32_F20X_F21X_PORT_H_
#define STM32F2_API_STM32_F20X_F21X_PORT_H_
	
#endif
 \end{lstlisting}
	
	\item После \textit{define} строки защиты следует пустая строка, за которой располагается \textit{include} на файл конфигурации библиотеки, имеющий имя \textbf{perfix\_conf.h} (название библиотеки выступает в качестве префикса).\\\textbf{Пример:}\begin{lstlisting}[language=C++, frame=tlBR, basicstyle=\fontsize{10}{10}\ttfamily]
#define STM32F2_API_STM32_F20X_F21X_PORT_H_
	
#include "stm32_f20x_f21x_conf.h"\end{lstlisting}
	
	\item В случае, если файл стоит включать в процесс компиляции только при каком-то условии, это условие (обернутое в \textit{ifdef}) необходимо указать через одну пустую строку после \textit{include} файла конфигурации библиотеки. Блок \textit{endif}, закрывающий тело блока условной компиляции должен быть написан без пустой строки перед \textit{endif}, закрывающим блок защиты повторной компиляции.\\\textbf{Пример:}\begin{lstlisting}[language=C++, frame=tlBR, basicstyle=\fontsize{10}{10}\ttfamily]
#ifndef STM32F2_API_STM32_F20X_F21X_PORT_H_
#define STM32F2_API_STM32_F20X_F21X_PORT_H_
	
#include "stm32_f20x_f21x_conf.h"
	
#ifdef MODULE_PORT
	
CODE
	
#endif
#endif
 \end{lstlisting}
	\item Внутри всех необходимых обёрток, описанных выше (на месте слова <<CODE>> примера из предыдущего пункта), располагается основное содержимое, уникальное для каждого типа .h файла:\begin{itemize}
		\item \textit{perfix\_moduleName.h} (дел~\ref{p:modul:h});
		\item \textit{perfix\_moduleName\_constexpr\_func.h} (подраздел~\ref{p:conf:h});
		\item \textit{perfix\-\_moduleName\-\_constexpr\-\_func\-\_class\-\_className.h} (подраздел~\ref{p:conf:ch});
		\item \textit{perfix\_moduleName\_struct.h} и \textit{perfix\-\_moduleName\-\_struct\-\_class\-\_class\-Name.h}\\(подраздел~\ref{p:struc:h});
	\end{itemize}
\end{enumerate}

\section{Содержимое perfix\_moduleName.h файла}\label{p:modul:h}
\begin{enumerate}
	\item include \textit{perfix\_moduleName\_struct.h} и \textit{perfix\_moduleName\_constexpr\_func.h}  файлов, если таковые имеются;
	\item пустая строка;
	\item краткое описание всего модуля, включающее в себя описание всех классов модуля, обернутое в многострочный комментарий с явно обозначенными границами символами <<*>> в количестве 70 штук;
	\item пустая строка;
	\item краткое описание класса, обернутое в много строчный комментарий;
	\item пустая строка;
	\item include \textit{perfix\_moduleName\_struct\_class\_className.h}, если имеется;
	\item пустая строка;
	\item тело класса;
	\item пустая строка;
	\item include файла \textit{perfix\-\_moduleName\-\_constexpr\-\_func\-\_class\-\_className.h}, если имеется;
	\item пустая строка;
	\item пункты 5-12 повторить для всех требуемых классов;
	\item пустая строка.
\end{enumerate}
\textbf{Пример всего файла:}\begin{lstlisting}[language=C++, frame=tlBR, basicstyle=\fontsize{10}{10}\ttfamily]
#ifndef STM32F2_API_STM32_F20X_F21X_PORT_H_
#define STM32F2_API_STM32_F20X_F21X_PORT_H_

#include "stm32_f20x_f21x_conf.h"

#ifdef MODULE_PORT

#include "stm32_f20x_f21x_port_struct.h"						
#include "stm32_f20x_f21x_port_constexpr_func.h"				

/**********************************************************************
 * Краткое описание модуля...
 **********************************************************************/

/*
 * Краткое описание класса pin...
 */

#include "stm32_f20x_f21x_port_struct_class_pin.h"

class pin {
public:
	...

private:
	...
};

#include "stm32_f20x_f21x_port_constexpr_func_class_pin.h"

/*
 * Краткое описание класса global_port...
 */

#include "stm32_f20x_f21x_port_struct_class_global_port.h"		

class global_port {
public:
	...
private:
	...
};

#include "stm32_f20x_f21x_port_constexpr_func_class_global_port.h"

#endif
#endif
 \end{lstlisting}

\section{Содержимое perfix\-\_module\-Name\-\_const\-expr\-\_func.h файла}\label{p:conf:h}
\begin{enumerate}
	\item include \textit{perfix\-\_moduleName\-\_struct.h} файла, если таковой имеются;
	\item пустая строка;
	\item краткое описание constexpr функции, обернутое в много строчный комментарий;
	\item тело функции;
	\item пустая строка;
	\item пункты 3-5 повторить для всех имеющихся методов;
\end{enumerate}
\textbf{Пример всего файла:}\begin{lstlisting}[language=C++, frame=tlBR, basicstyle=\fontsize{10}{10}\ttfamily]
#ifndef STM32F2_API_PORT_STM32_F20X_F21X_PORT_CONSTEXPR_FUNC_H_
#define STM32F2_API_PORT_STM32_F20X_F21X_PORT_CONSTEXPR_FUNC_H_

#include "stm32_f20x_f21x_conf.h"

#ifdef MODULE_PORT

#include "stm32_f20x_f21x_port_struct.h"

/*
 * Краткое описание constexpr функции p_base_port_address_get...
 */

constexpr uint32_t p_base_port_address_get( EC_PORT_NAME port_name ) {
	CODE;
}

/*
 * Краткое описание constexpr функции bb_p_port_look_key_get...
 */
constexpr uint32_t bb_p_port_look_key_get ( EC_PORT_NAME port_name ) {
	CODE;
}

#endif
#endif
 \end{lstlisting}

\section{Содержимое perfix\-\_moduleName\-\_constexpr\-\_func\-\_class\-\_class-\\\-Name.h файла}\label{p:conf:ch}
\begin{enumerate}
	\item include \textit{perfix\_moduleName\_struct.h} и \textit{perfix\_moduleName\_struct\_class\_className.h} файлов, если таковые имеются;
	\item пустая строка;
	\item комментарий о начале области с constexpr конструктором(-ами), обернутый в многострочный комментарий с явно обозначенными границами символами <<*>> в количестве 70 штук. После комментария должна следовать  пустая строка;
	\item тело constexpr конструктора;
	\item пустая строка;
	\item пункты 3-5 повторяются для всех имеющихся конструкторов;
	\item комментарий о начале области с constexpr методами, , обернутый в много строчный комментарий с явно прописанными границами, бросающимися в газа;
	\item пустая строка;
	\item краткое описание constexpr функции, обернутое в много строчный комментарий;
	\item тело функции;
	\item пустая строка;
	\item пункты 9-11 повторить для всех имеющихся методов;
	
\end{enumerate}
\textbf{Пример всего файла:}\begin{lstlisting}[language=C++, frame=tlBR, basicstyle=\fontsize{10}{10}\ttfamily]
#ifndef STM32F2_API_PORT_STM32_F20X_F21X_PORT_CONSTEXPR_FUNC_CLASS_PIN_H_
#define STM32F2_API_PORT_STM32_F20X_F21X_PORT_CONSTEXPR_FUNC_CLASS_PIN_H_

#include "stm32_f20x_f21x_conf.h"

#ifdef MODULE_PORT

#include "stm32_f20x_f21x_port_struct.h"
#include "stm32_f20x_f21x_port_struct_class_pin.h"

/**********************************************************************
 * Область constexpr конструкторов.
 **********************************************************************/
constexpr pin::pin ( const pin_config_t* pin_cfg_array ):
	КОД ИНИЦИАЛИЗАЦИИ ПЕРЕМЕННЫХ КЛАССА {};

/**********************************************************************
 * Область constexpr функций.
 **********************************************************************/
 
/*
 * Краткое описание constexpr функции set_msk_get...
 */
constexpr uint32_t pin::set_msk_get ( const pin_config_t* const pin_cfg_array ) {
	CODE;
}

/*
 * Краткое описание constexpr функции p_base_port_address_get...
 */
constexpr uint32_t pin::p_base_port_address_get ( const pin_config_t* const 
																									pin_cfg_array ) {
	CODE;
}

#endif
#endif
 \end{lstlisting}

\section{Содержимое perfix\_moduleName\_struct.h и\\erfix\_moduleName\_struct\_class\_className.h файлов}\label{p:struc:h}
В файлах располагаются следующие конструкции (сверху вниз):
\begin{enumerate}
	\item Enum class-ы.
	\item Макросы.
	\item Packed структуры.
	\item Структуры.
\end{enumerate}
Правила оформления следующие:
\begin{enumerate}
	\item Перед каждым из выше перечисленных пунктов размещается комментарий о начале области конкретного пункта. Он оборачивается в многострочный комментарий с явно обозначенными границами. Граница представляют из себя строку символов <<*>>. Верхняя и нижняя строка должна содержать 70 разграничивающих символов. После выделенного комментария следует пустая строка.
	\item Перед каждым элементом пункта размещается краткое описание элемента, обернутое в много строчный комментарий. После краткого описания пустая строка не ставится.
	\item Между предыдущим элементом пункта и комментарием следующего элемента этого же пункта вставляется пустая строка.
	\item Между предыдущим элементом пункта и комментария области следующего пункта вставляется пустая строка.
\end{enumerate}
\textbf{Пример всего файла:}\begin{lstlisting}[language=C++, frame=tlBR, basicstyle=\fontsize{10}{10}\ttfamily]
#ifndef STM32F2_API_PORT_STM32_F20X_F21X_PORT_STRUCT_CLASS_PIN_H_
#define STM32F2_API_PORT_STM32_F20X_F21X_PORT_STRUCT_CLASS_PIN_H_

#include "stm32_f20x_f21x_conf.h"

#ifdef MODULE_PORT

/**********************************************************************
 * Область упакованных структур.
 **********************************************************************/
 
/*
 * Краткое описание упакованной структуры.
 */
struct __attribute__((packed)) port_registers_struct {
	ПОЛЯ СТРУКТУРЫ;
};

/**********************************************************************
 * Область структур.
 **********************************************************************/

/*
 * Краткое описание структуры.
 */
struct st_struct {
	ПОЛЯ СТРУКТУРЫ;
};

/**********************************************************************
 * Область enum class-ов.
 **********************************************************************/
 
/*
 * Краткое описание num class-а.
 */
enum class EC_PORT_PIN_NAME {
	VALUE_FIELDS;
};


/**********************************************************************
 * Область макросов.
 **********************************************************************/

/*
 * Краткое описание макроса.
 */
#define M_PIN_CFG_ADC(PORT,PIN)	{
	VALUE_FIELDS;
}

#endif
#endif
 \end{lstlisting}						% Оформление .h файлов библиотеки.
\section{Объявление классов в .h файлах}\label{class:0}
\begin{enumerate}
	\item Общие сведения об оформлении class-ов в .h файлах (подраздел~\ref{OBK}).
	\item В public области должны располагаться (с соблюдением последовательности сверху вниз):\begin{itemize}
		\item конструктор(-ы) класса (подраздел~\ref{K:0:0});
		\item constexpr конструктор(-ы) класса (подраздел~\ref{K:0:1});
		\item constexpr методы класса (подраздел~\ref{constexpr:0});
		\item доступные пользователю нестатические методы класса, выполняющиеся в реальном времени и возвращающие значение стандартного типа или указатель на стандартный тип данных (подраздел~\ref{dp:n:s});
		\item доступные пользователю нестатические методы класса, выполняющиеся в реальном времени и возвращающие значение нестандартного типа или указатель на нестандартный тип данных (подраздел~\ref{dp:n:n};
		\item доступные пользователю статические (static) методы класса, выполняющиеся в реальном времени и возвращающие значение стандартного типа или указатель на стандартный тип (подраздел~\ref{dp:s:s});
		\item доступные пользователю статические (static) методы класса, выполняющиеся в реальном времени и возвращающие значение нестандартного типа или указатель на нестандартный тип (подраздел~\ref{dp:s:n});
		\item открытие переменные и константы класса, доступные пользователю напрямую (подраздел~\ref{dp:op}).
	\end{itemize}
	\item В private область должны располагаться (с соблюдением последовательности сверху вниз):\begin{itemize}
		\item внутренние constexpr методы класса, возвращающие значение стандартного типа или указатель на стандартный тип данных (подраздел~\ref{zp:constexpr:s});
		\item внутренние constexpr методы класса, возвращающие значение нестандартного типа или указатель на нестандартный тип данных (подраздел~\ref{zp:constexpr:n});
		\item закрытые нестатические методы класса, выполняющиеся в реальном времени и возвращающие значение стандартного типа или указатель на стандартный тип данных (подраздел~\ref{zp:n:s});
		\item закрытые нестатические методы класса, выполняющиеся в реальном времени и возвращающие значение нестандартного типа или указатель на нестандартный тип данных (подраздел~\ref{zp:n:n});
		\item закрытые статические (static) методы класса, выполняющиеся в реальном времени и возвращающие значение стандартного типа или указатель на стандартный тип (подраздел~\ref{zp:s:s});
		\item закрытые статические (static) методы класса, выполняющиеся в реальном времени и возвращающие значение нестандартного типа или указатель на нестандартный тип (подраздел~\ref{zp:s:n});
		\item закрытые константы класса стандартных типов (подраздел~\ref{zp:const:s}).
		\item закрытые константы класса нестандартных типов (подраздел~\ref{zp:const:n}).
		\item закрытые переменные класса стандартных типов (подраздел~\ref{zp:pp:s}).
		\item закрытые переменные класса нестандартных типов (подраздел~\ref{zp:pp:n}).
	\end{itemize}
\end{enumerate}

\subsection{Общие сведения об оформлении class-ов в .h файлах}\label{OBK}
\begin{itemize}
	\item Между зарезервированным словом class и именем класса ставится один (1) пробел.
	\item Между последним символом имени класса и открывающейся фигурной скобкой ставится один (1) пробел.
	\item Сначала идет public, а за ним private область.
	\item <<\}>> (скобка закрывающая тело класса) должна находится на новой строке.
\end{itemize}\begin{lstlisting}[language=C++,frame=tlBR]
class name_class {
public:
private:
};\end{lstlisting}

\subsection{Конструктор(-ы) класса}\label{K:0:0}
Использование не constexpr конструкторов классов запрещено. Это связано с неочевидной последовательностью вызова конструкторов глобальных объектов, которая может привести к неверной инициализации объекта (если явно не указывать последовательность вызовов с помощью специальных директив компоновщика). Например, сначала будет предпринята попытка инициализировать внешнюю переферию (за пределами микроконтроллера), не инициализировав интерфейс, по которому она подключена.

В случае если пользователь все же создаст объект, конструктор которого будет требовать выполнения кода функции конструктора во время инициализации, вызов его метода инициализации произведен не будет (объект останется не инициализированным).

\subsection{Constexpr конструктор(-ы) класса}\label{K:0:1}
\begin{itemize}
	\item В случае, если конструкторов несколько, они должны быть расположены от большего количества входных параметров к меньшему.
	\item Реализация самого конструктора не должна находится в теле класса. Ее (реализацию конструктора) следует вынести в отдельный файл.
	\item Перед словом constexpr должен быть выполнен отступ в 1 tab.	
	\item Между словом constexpr и именем конструктора(-ов) ставится один (1) пробел.
	\item После имени конструктора должен быть выполнен один (1) пробел. 
	\item Аргументы конструктора(-ов) в скобках должны быть разделены <<, >> (запятая + пробел).
	\item Внутри скобок перечисления аргументов конструктора должен быть отступ в 1 пробел с каждой стороны.\\\textbf{Пример: } <<( uint32\_t a, uint8\_t b )>>.
\end{itemize}
\textbf{Пример:}\begin{lstlisting}[language=C++, frame=tlBR, basicstyle=\fontsize{8}{8}\ttfamily]
	constexpr pin ( const pin_config_t *pin_cfg_array, const uint32_t pin_cout );
	constexpr pin ( const pin_config_t *pin_cfg_array );
\end{lstlisting}

\subsection{Constexpr методы класса}\label{constexpr:0}
Размещение constexpr методов в разделе public запрещено и не имеет смысла.

\subsection{Доступные пользователю нестатические методы класса, выполняющиеся в реальном времени и возвращающие значение стандартного типа или указатель на стандартный тип данных}\label{dp:n:s}
\begin{itemize}
	\item Перед типом возвращаемого значения должен быть выполнен отступ в один (1) tab.
	\item Имена методов должны быть выравнены с помощью tab с остальными методами этого типа. Выравнивание методов других типов производится по иной сетке.
	\item Аргументы методов в скобках должны быть разделены <<, >> (запятая + пробел).
	\item Внутри скобок перечисления аргументов метода должен быть отступ в один (1) пробел с каждой стороны.\\\textbf{Пример: } <<( uint32\_t a, uint8\_t b )>>.
	\item В случае, если метод не изменяет данные класса, после параметров в скобках следует поставить один (1) пробел, после чего слово <<const;>>. <<;>> закрывает заголовок функции.
\end{itemize}
\textbf{Пример:}\begin{lstlisting}[language=C++, frame=tlBR, basicstyle=\fontsize{8}{8}\ttfamily]
	void	set		( void ) const;
	void	reset		( void ) const;
	void	invert		( void ) const;
	int	read		( void ) const;
\end{lstlisting}

\subsection{Доступные пользователю нестатические методы класса, выполняющиеся в реальном времени и возвращающие значение нестандартного типа или указатель на нестандартный тип данных}\label{dp:n:n}
\begin{itemize}
	\item В качестве нестандартного типа может выступать enum class или структура.
	\item В качестве указателя на нестандартный тип может выступать указатель на enum class переменную или структуру.
	\item Перед возвращаемым типом метода должен быть отступ в один (1) tab.
	\item Имена методов должны быть выравнены с помощью tab с остальными методами этого типа. Выравнивание методов других типов производится по иной сетке.
	\item Аргументы методов в скобках должны быть разделены <<, >> (запятая + пробел).
	\item Внутри скобок перечисления аргументов метода должен быть отступ в 1 пробел с каждой стороны.\\\textbf{Пример: } <<( uint32\_t a, uint8\_t b )>>.
\end{itemize}
\textbf{Пример:}\begin{lstlisting}[language=C++, frame=tlBR, basicstyle=\fontsize{8}{8}\ttfamily]
	E_ANSWER_GP	met_g	( void );
\end{lstlisting}

\subsection{Доступные пользователю статические (static) методы класса, выполняющиеся в реальном времени и возвращающие значение стандартного типа или указатель на стандартный тип}\label{dp:s:s}	\begin{itemize}
	\item Перед зарезервированным словом <<static>> должен быть отступ в один (1) tab.
	\item Между <<static>> и типом возвращаемого значения должен быть выполнен отступ в один (1) пробел.
	\item Имена методов должны быть выравнены с помощью tab с остальными методами этого типа. Выравнивание методов других типов производится по иной сетке.
	\item Аргументы методов в скобках должны быть разделены <<, >> (запятая + пробел).
	\item Внутри скобок перечисления аргументов метода должен быть отступ в 1 пробел с каждой стороны.\\\textbf{Пример: } <<( uint32\_t a, uint8\_t b )>>.
	\item Так как предполагается, что данный метод будет работать с объектом(-ами) класса, в котором(-ых) описан его заголовок, то первым аргументом метода должен быть указатель на void, который внутри класса будет разыменован в указатель на объект класса, в котором был объявлен.\\Используется указатель на void, а не на объект класса с целью совместимости с FreeRTOS, написанной на C.
	\item Первый аргумент метода следует называть <<void *obj>>.
\end{itemize}
\textbf{Пример:}\begin{lstlisting}[language=C++, frame=tlBR, basicstyle=\fontsize{8}{8}\ttfamily]
	static void	task_1	( void *obj );
	static int	m_1	( void *obj );
\end{lstlisting}

\subsection{Доступные пользователю статические (static) методы класса, выполняющиеся в реальном времени и возвращающие значение нестандартного типа или указатель на нестандартный тип}\label{dp:s:n}
\begin{itemize}
	\item В качестве нестандартного типа может выступать enum class или структура.
	\item В качестве указателя на нестандартный тип может выступать указатель на enum class переменную или структуру.
	\item Перед зарезервированным словом <<static>> должен быть отступ в один (1) tab.
	\item Между <<static>> и типом возвращаемого значения должен быть выполнен отступ в один (1) пробел.
	\item Имена методов должны быть выравнены с помощью tab с остальными методами этого типа. Выравнивание методов других типов производится по иной сетке.
	\item Аргументы методов в скобках должны быть разделены <<, >> (запятая + пробел).
	\item Внутри скобок перечисления аргументов метода должен быть отступ в 1 пробел с каждой стороны.\\\textbf{Пример: } <<( uint32\_t a, uint8\_t b )>>.
	\item Так как предполагается, что данный метод будет работать с объектом(-ами) класса, в котором(-ых) описан его заголовок, то первым аргументом метода должен быть указатель на void, который внутри класса будет разыменован в указатель на объект класса, в котором был объявлен.\\Используется указатель на void, а не на объект класса с целью совместимости с FreeRTOS, написанной на C.
	\item Первый аргумент метода следует называть <<void *obj>>.
\end{itemize}
\textbf{Пример:}\begin{lstlisting}[language=C++, frame=tlBR, basicstyle=\fontsize{8}{8}\ttfamily]
	static E_ANSWER_GP	met_a	( void *obj );
\end{lstlisting}

\subsection{Открытие переменные и константы класса, доступные пользователю напрямую}\label{dp:op}
Размещение переменных (изменяемых или заданных как const) в public области запрещено. Даже в случае, если требуется просто читать/записывать одну переменную, следует сделать отдельный метод(-ы) для этого. Так как прямое чтение данных из класса является нарушением ООП.

\subsection{Внутренние constexpr методы класса, возвращающие значение стандартного типа или указатель на стандартный тип данных}\label{zp:constexpr:s}
\begin{itemize}
	\item Реализация тела функции не должна находится в теле класса. Она (реализация тела функции) должна быть вынесена в отдельный файл. Допускаются только заголовки функций.
	\item Перед словом constexpr должен быть выполнен отступ в один (1) tab.
	\item Между зарезервированным словом constexpr и типом возвращаемого значения требуется поставить один (1) пробел.
	\item Имена методов должны быть выравнены с помощью tab с остальными методами этого типа. Выравнивание методов других типов производится по иной сетке.
	\item Аргументы методов в скобках должны быть разделены <<, >> (запятая + пробел).
	\item Внутри скобок перечисления аргументов метода должен быть отступ в 1 пробел с каждой стороны.\\\textbf{Пример: } <<( uint32\_t a, uint8\_t b )>>.
\end{itemize}
\textbf{Пример:}\begin{lstlisting}[language=C++, frame=tlBR, basicstyle=\fontsize{8}{8}\ttfamily]
	constexpr uint32_t	moder_reg_reset_init_msk_get	( EC_PORT_NAME port_name );
\end{lstlisting}

\subsection{Внутренние constexpr методы класса, возвращающие значение нестандартного типа или указатель на нестандартный тип данных}\label{zp:constexpr:n}
\begin{itemize}
	\item В качестве нестандартного типа может выступать enum class или структура.
	\item В качестве указателя на нестандартный тип может выступать указатель на enum class переменную или структуру.
	\item Реализация тела функции не должна находится в теле класса. Она (реализация тела функции) должна быть вынесена в отдельный файл. Допускаются только заголовки функций.
	\item Перед словом constexpr должен быть выполнен отступ в один (1) tab.
	\item Между зарезервированным словом constexpr и типом возвращаемого значения требуется поставить один (1) пробел.
	\item Имена методов должны быть выравнены с помощью tab с остальными методами этого типа. Выравнивание методов других типов производится по иной сетке.
	\item Аргументы методов в скобках должны быть разделены <<, >> (запятая + пробел).
	\item Внутри скобок перечисления аргументов метода должен быть отступ в 1 пробел с каждой стороны.\\\textbf{Пример: } <<( uint32\_t a, uint8\_t b )>>.
\end{itemize}\begin{lstlisting}[language=C++, frame=tlBR, basicstyle=\fontsize{8}{8}\ttfamily]
	constexpr EC_PORT_NAME	port_name_get	( uint32_t value_reg );
\end{lstlisting}

\subsection{Закрытые нестатические методы класса, выполняющиеся в реальном времени и возвращающие значение стандартного типа или указатель на стандартный тип данных}\label{zp:n:s}
Оформляются так же, как и открытые нестатические методы класса, выполняющиеся в реальном времени и возвращающие значение стандартного типа или указатель на стандартный тип данных (подраздел~\ref{dp:n:s}).

\subsection{Закрытые нестатические методы класса, выполняющиеся в реальном времени и возвращающие значение нестандартного типа или указатель на нестандартный тип данных}\label{zp:n:n}
Оформляются так же, как и открытые нестатические методы класса, выполняющиеся в реальном времени и возвращающие значение нестандартного типа или указатель на нестандартный тип данных (подраздел~\ref{dp:n:n}).

\subsection{Закрытые статические (static) методы класса, выполняющиеся в реальном времени и возвращающие значение стандартного типа или указатель на стандартный тип}\label{zp:s:s}
Оформляются так же, как и открытые статические (static) методы класса, выполняющиеся в реальном времени и возвращающие значение стандартного типа или указатель на стандартный тип (подраздел~\ref{dp:s:s}).

\subsection{Закрытые статические (static) методы класса, выполняющиеся в реальном времени и возвращающие значение нестандартного типа или указатель на нестандартный тип}\label{zp:s:n}
Оформляются так же, как и открытые статические (static) методы класса, выполняющиеся в реальном времени и возвращающие значение нестандартного типа или указатель на нестандартный тип (подраздел~\ref{dp:s:n}).

\subsection{Закрытые константы класса стандартных типов}\label{zp:const:s}
\begin{itemize}
	\item На одной строке допустимо объявлять лишь одну константу.
	\item Перед зарезервированным словом <<const>> должен быть поставлен один (1) tab.
	\item Имена констант должны быть выравнены с помощью tab с остальными константами этого типа. Выравнивание констант других типов производится по иной сетке.
\end{itemize}\begin{lstlisting}[language=C++, frame=tlBR, basicstyle=\fontsize{8}{8}\ttfamily]
	const uint32_t	count;
\end{lstlisting}

\subsection{Закрытые константы класса нестандартных типов}\label{zp:const:n}
Оформляются так же, как и закрытые константы класса стандартных типов (подраздел~\ref{zp:const:s})\begin{lstlisting}[language=C++, frame=tlBR, basicstyle=\fontsize{8}{8}\ttfamily]
	const global_port_msk_reg_struct	gb_msk_struct;
\end{lstlisting}

\subsection{Закрытые переменные класса стандартных типов}\label{zp:pp:s}
\begin{itemize}
	\item На одной строке допустимо объявлять лишь одну переменную.
	\item Перед типом должен быть поставлен один (1) tab.
	\item Имена переменных должны быть выравнены с помощью tab с остальными переменными этого типа. Выравнивание переменных других типов производится по иной сетке.
\end{itemize}\begin{lstlisting}[language=C++, frame=tlBR, basicstyle=\fontsize{8}{8}\ttfamily]
	uint32_t	flag;
\end{lstlisting}

\subsection{Закрытые переменные класса нестандартных типов}\label{zp:pp:n}
Оформляются так же, как и закрытые переменные класса стандартных типов (подраздел~\ref{zp:pp:s})\begin{lstlisting}[language=C++, frame=tlBR, basicstyle=\fontsize{8}{8}\ttfamily]
	EC_FL	mb_msk_struct;
\end{lstlisting}						% Объявление классов в .h файлах.
\section{Объявление packed структур}\label{struct:p}
\subsection{Когда стоит объявлять структуру как packed?}
Структуру следует объявить как packed, если она:
\begin{itemize}
	\item описывает совокупность регистров аппаратного блока периферии (у которого, как известно, регистры имеют четко фиксированный размер и порядок следования);
	\item описывает образ памяти регистров аппаратного блока;
	\item описывает структуру какого-либо пакета со строго фиксированными полями.
\end{itemize}

\subsection{Размещение packed структур}
Packed структуры должны быть размещены только в \textbf{.h} файлах. 

\subsection{Оформление packed структур}
\begin{itemize}
	\item перед первой объявленной packed структурой размещается комментарий о начале области packed структур, обернутый в много строчный комментарий с явно обозначенными границами, бросающимися в газа. После чего вставляется пустая строка;
	\item перед каждой packed структурой размещается ее краткое описание, обернутое в много строчный комментарий. После краткого описания пустая строка не ставится;
	\item заголовок структуры следует оформить следующим образом:
	\begin{enumerate}
		\item ключевое слово struct без отступов в начале строки;
		\item отступ в один (1) пробел;
		\item директива препроцессора <<\_\_attribute\_\_((packed))>>;
		\item пробел;
		\item имя структуры;
		\item пробел;
		\item открывающая тело packed структуры скобка <<\{>>;
	\end{enumerate}
	\item поля структуры следует оформлять следующим образом:
	\begin{enumerate}
		\item каждая строка начинается с отступа в один (1) tab;
		\item ключевое слово volatile;
		\item один (1) пробел;
		\item тип поля;
		\item требуемое количество отступов, выполненных с помощью tab;
		\item имя поля;
		\item <<;>>;
		\item требуемое количество tab;
		\item <<// >> (// + пробел) + одно строчный комментарий.
	\end{enumerate}
	\item все имена полей структуры должны быть выравнены с помощью tab между собой;
	\item после последнего поля структуры следует скобка закрытия тела структуры (<<\}>>);
	\item после последней packed структуры вставляется пуста строка.
\end{itemize}\textbf{Пример packed области:}\begin{lstlisting}[language=C++, frame=tlBR, basicstyle=\fontsize{9}{9}\ttfamily]
/**************************************************************
 * Packed struct label.
 **************************************************************/

/*
 * Packed struct description...
 */
struct __attribute__((packed)) port_registers_struct {
	volatile uint32_t	mode;		// Comment...
	volatile uint32_t	otype;		// Comment...
	...;
};

\end{lstlisting}					% Объявление packed структур.
\chapter{Объявление структур}\label{struct}
\section{Когда стоит оборачивать данные в структуру?}
Данные следует обернуть в структуру, если:
\begin{itemize}
	\item требуется передавать более одного параметра в конструктор класса;
	\item требуется вернуть из функции более одного параметра;
\end{itemize}
\textbf{Замечание: }перед тем, как оборачивать данные в структуру проверьте, не имеются ли у вас условий, согласно которым данные должны быть обернуты в упакованную структуру (раздел~\ref{struct:p})
\section{Размещение структур}
Прототипы структур должны быть размещены только в \textbf{.h} файлах. Экземпляры - в \textbf{.cpp}.

\section{Оформление структур}
\begin{itemize}
	\item перед первой объявленной структурой размещается комментарий о начале соответствующей области (области структур), обернутый в многострочный комментарий с явно обозначенными границами символами <<*>> в количестве 70 штук. После комментария должна следовать  пустая строка;
	\item перед каждой структурой размещается ее краткое описание, обернутое в многострочный комментарий. После краткого описания пустая строка не ставится;
	\item заголовок структуры следует оформить следующим образом:
	\begin{enumerate}
		\item ключевое слово struct без отступов в начале строки;
		\item отступ в один (1) пробел;
		\item имя структуры;
		\item пробел;
		\item открывающая тело packed структуры скобка <<\{>>;
	\end{enumerate}
	\item поля структуры следует оформлять следующим образом:
	\begin{enumerate}
		\item каждая строка начинается с отступа в один (1) tab;
		\item тип поля;
		\item требуемое количество отступов, выполненных с помощью tab;
		\item имя поля;
		\item <<;>>;
		\item требуемое количество tab;
		\item <<// >> (// + пробел) + одно строчный комментарий.
	\end{enumerate}
	\item все имена полей структуры должны быть выравнены с помощью tab между собой;
	\item после последнего поля структуры следует скобка закрытия тела структуры (<<\}>>);
	\item после последней структуры вставляется пуста строка.
\end{itemize}\textbf{Пример области структур:}\begin{lstlisting}[language=C++, frame=tlBR, basicstyle=\fontsize{10}{10}\ttfamily]
/**********************************************************************
 * Область структур.
 **********************************************************************/

/*
 * Краткое описание структуры...
 */
struct a {
	uint32_t	b;		// Пояснение к полю b.
	uint32_t	c;		// Пояснение к полю c.
};\end{lstlisting}							% Объявление структур.
\chapter{Объявление enum class-ов}\label{ec:h:0}
\section{Когда стоит объявлять enum class?}
Enum class-ы стоит объявлять, если какому-то полю структуры/переменной требуется присвоить какое-то одно значение из заранее известного ряда, каждому из которых можно присвоить уникальное имя.

\section{Размещение enum class-ов}
Прототипы enum class-ов должны быть размещены только в \textbf{.h} файлах.

\section{Оформление enum class-ов}
\begin{itemize}
	\item перед первым объявленным enum class-ом размещается комментарий о начале соответствующей области (области enum class-ов), обернутый в многострочный комментарий с явно обозначенными границами символами <<*>> в количестве 70 штук. После комментария должна следовать  пустая строка;
	\item перед каждым enum class-ом размещается его краткое описание, обернутое в многострочный комментарий. После краткого описания пустая строка не ставится;
	\item заголовок enum class-а следует оформить следующим образом:
	\begin{enumerate}
		\item ключевое словосочетание <<enum class>> без отступов в начале строки;
		\item отступ в один (1) пробел;
		\item имя enum class-а (в соответствии с правилами написания имен enum class-ов);
		\item пробел;
		\item открывающая тело enum class-а скобка <<\{>>;
	\end{enumerate}
	\item значения enum class-ов следует оформлять следующим образом:
	\begin{enumerate}
		\item каждое значение начинается с отступа в один (1) tab;
		\item имя значения;
		\item требуемое количество отступов, выполненных с помощью tab;
		\item <<=	>> (<<=>> + один (1) tab);
		\item непосредственное цифровое значение;
		\item <<,>> (в случае, если элемент последний, запятая не ставится).
		\item требуемое количество tab;
		\item <<// >> (// + пробел) + однострочный комментарий.
	\end{enumerate}
	\item выравнивание производится по знаку <<=>> с левой стороны;
	\item после последнего значения enum class-а следует скобка закрытия его тела и точка с запятой (<<\};>>), расположенная на новой строке;
	\item после последнего enum class - а вставляется пуста строка.
\end{itemize}\textbf{Пример области структур:}\begin{lstlisting}[language=C++, frame=tlBR, basicstyle=\fontsize{10}{10}\ttfamily]
/**********************************************************************
 * Область enum class- ов.
 **********************************************************************/

enum class EC_PORT_NAME {
	A	=	0,
	B	=	1,
	C	=	2,
	D	=	3,
	H	=	4
};\end{lstlisting}							% Объявление enum class-ов.
\chapter{Объявление макросов}\label{mackros:0}
\section{Когда стоит объявлять макрос?}
Макрос следует объявить, если имеется фрагмент кода, многократно повторяющийся в коде программы с незначительными изменениями в каждом конкретном случае. При этом использовать отдельную constexpr функцию нерационально или невозможно.\\Пример: при заполнении массива структур инициализации выводов некоторое количество выводов инициализируются как входы ADC. Для того, чтобы не писать каждый раз все параметры каждого пина, достаточно будет воспользоваться макросом, в котором нужно указать изменяющиеся параметры: имя порта и вывода.

\section{Размещение макросов}
Прототипы макросов должны быть размещены в \textbf{.h} файлах.

\section{Оформление макросов}
\begin{itemize}
	\item перед первым объявленным макросом размещается комментарий о начале соответствующей области (области макросов), обернутый в многострочный комментарий с явно обозначенными границами символами <<*>> в количестве 70 штук. После комментария должна следовать  пустая строка;
	\item перед каждым макросом размещается его краткое описание, обернутое в многострочный комментарий. После краткого описания пустая строка не ставится;
	\item заголовок макроса следует оформить следующим образом:
	\begin{enumerate}
		\item ключевое слово \#define без отступов в начале строки;
		\item отступ в один (1) пробел;
		\item имя макроса (в соответствии с правилами написания макросов);
		\item открывающаяся скобка (<<(>>);
		\item параметры макроса через запятую без пробелов в соответствии с правилами написания макросов;
		\item закрывающаяся скобка (<<)>>);
		\item тело макроса;
	\end{enumerate}
\end{itemize}\textbf{Пример:}\begin{lstlisting}[language=C++, frame=tlBR, basicstyle=\fontsize{10}{10}\ttfamily]
/**********************************************************************
 * Область макросов.
 **********************************************************************/

/*
 * Возвращает структуру конфигурации вывода,
 * настроенного на вход, подключенный к ADC.
 */
#define M_PIN_CFG_ADC(PORT,PIN)	{											\
	.port							= PORT,														\
	.pin_name					= PIN,														\
	.mode							= EC_PIN_MODE::ANALOG,						\
	.output_config		= EC_PIN_OUTPUT_CFG::NOT_USE,			\
	.speed						= EC_PIN_SPEED::LOW,							\
	.pull							= EC_PIN_PULL::NO,								\
	.af								= EC_PIN_AF::NOT_USE,							\
	.locked						= EC_LOCKED::LOCKED,							\
	.state_after_init	= EC_PIN_STATE_AFTER_INIT::NO_USE	\
}\end{lstlisting}						% Объявление макросов.

\if 0
\fi		% Соглашение о написании и оформлении библиотеки.
\part{СОГЛАШЕНИЕ О НАПИСАНИИ И ОФОРМЛЕНИИ БИБЛИОТЕКИ}\label{logic:bib2}
\subimport	{./}	{introduction.tex}								% Введение.
\chapter{Средства сборки}\label{compgcc:0}
В основе библиотеки лежат constexpr функции, полноценная поддержка которых появилась в C++14. Отсюда следует вывод, что минимально возможная версия используемого языка - C++14. В случае, если в более поздних версиях будет несовместимость с C++14, следует внести изменения в библиотеку, решающие вопросы несовместимости по средствам проверки версии используемого стандарта языка и выбора совместимого с ним участка кода.

Для компиляции библиотеки следует использовать arm-none-eabi-g++ не старее (GNU Tools for ARM Embedded Processors 6-2017-q1-update) 6.3.1 20170215 (release) [ARM/embedded-6-branch revision 245512].				% Средства сборки.
\section{Дерево проекта и именование файлов}\label{dn:0}
Правила, касающиеся оформления библиотеки:
\begin{enumerate}
	\item Для файлов, относящихся к работе с блоками аппаратной и программной (абстрактные) периферии, должна существовать своя папка на каждый модуль.\\
	Пример: \textit{rcc}, \textit{port}, \textit{pwr} и т.д.\\
	Имя папки должно содержать только название аппаратного модуля, написанного строчными буквами латинского алфавита.
	
	\item Каждая папка, посвящённая определённому блоку периферии (аппаратной или программной), должна содержать следующие файлы:
	\begin{itemize}
		\item \textbf{perfix\_moduleName.h}\\
		В данном файле должны находится классы, относящиеся к определённому блоку периферии. Объекты этих классов можно использовать в коде пользователя.
		\item \textbf{perfix\_moduleName.cpp}\\
		Если в perfix\_moduleName.h всего один класс, то в данном файле находятся методы класса из файла perfix\_moduleName.h, вызов которых производится в реальном времени.\\
		В случае, если классов несколько и у них нет static общих методов (используемые двумя и более классами) - данный файл создавать не следует. Вместо этого для уникальных методов каждого класса должен быть свой файл с соответствующим постфиксом (именем класса). Об этом ниже.
		\item \textbf{perfix\_moduleName\_class\_className.cpp}\\
		В случае, если в файле perfix\_moduleName.h более одного класса и какой-то из этих классов имеет методы, доступные только ему - их следует вынести в отдельный файл с постфиксом, соответствующим имени класса, к которому он (метод) относится.\\
		В случае, если в файле perfix\_moduleName.h один класс, методы, относящиеся к этому классу, должны быть размещены в файле perfix\_\-moduleName.cpp.
		\item \textbf{perfix\_moduleName\_constexpr\_func.h}\\
		В данном файле содержатся все constexpr методы, которые используются классом(-и) из файла perfix\_moduleName.h. Эти методы, как правило, являются private методами класса(-ов).\\
		В случае, если в файле perfix\_moduleName.h более одного класса, в данном файле должны находятся лишь те методы, которые используются всеми классами файла perfix\_moduleName.h.\\
		В случае, если каждый класс файла perfix\_moduleName.h использует лишь свой определенный набор методов, никак не пересекающийся с остальными классами, данный файл создавать не следует.
		\item \textbf{perfix\_moduleName\_constexpr\_func\_class\_className.h}\\
		В случае, если в файле perfix\_moduleName.h более одного класса и у какого-то из классов имеются constexpr методы, никак не связанные с остальными (используются только им), их следует вынести в отдельный файл.\\
		В случае, если таких классов несколько (каждый из которых использует свои определенные constexpr методы), то для каждого такого класса следует создать отдельный файл.
		\item \textbf{perfix\_moduleName\_struct.h}\\		
		В данном файле содержатся все структуры и enum class-ы, используемые всеми классами файла perfix\_moduleName.h.\\		
		В случае, если классы не имеют общих структур или enum class-ов, данный файл создавать не следует.\\		
		В случае, если в perfix\_moduleName.h всего один класс, его структуры и enum class-ы должны располагаться здесь без создания конкретного файла под конкретный класс (из пункта ниже).
		\item \textbf{perfix\_moduleName\_struct\_class\_className.h}\\		
		В случае, если классов в файле perfix\_moduleName.h более одного и у какого-то из классов имеются структуры или enum class-ы, которые используются только им одним, данные структуры и/или enum class-ы требуется вынести в отдельный файл с постфиксом имени класса, к которому они относятся.
	\end{itemize}

	Имена всех файлов должны быть написаны строчными латинскими символами (маленькие английские буквы). В том числе и сокращения по типу <<pwr>>.
	
	Все слова в имени должны разделяться символами нижнего подчеркивания. 
	
	В качестве примера рассмотрим дерево папки port библиотеки stm32\_f20x\_f21x (название библиотеки выступает в качестве префикса).
	
	stm32\_f20x\_f21x\_port.h содержит 2 класса (global\_port и pin). У них есть общие структуры, enum class-ы и методы. Однако есть и личные (используемые только ими) структуры, enum class-ы и constexpr методы. При этом у них нет общих static методов.
	\begin{lstlisting}[language=C++, frame=tlBR, basicstyle=\fontsize{10}{10}\ttfamily]
stm32_f20x_f21x_port_class_global_port.cpp
stm32_f20x_f21x_port_class_pin.cpp 
stm32_f20x_f21x_port_constexpr_func_class_global_port.h
stm32_f20x_f21x_port_constexpr_func_class_pin.h
stm32_f20x_f21x_port_constexpr_func.h
stm32_f20x_f21x_port_struct_class_global_port.h
stm32_f20x_f21x_port_struct_class_pin.h
stm32_f20x_f21x_port_struct.h
stm32_f20x_f21x_port.h\end{lstlisting}\end{enumerate}	% Дерево проекта и именование файлов.
\section{Принятые сокращения}\label{sk:0}
\begin{enumerate}
	\item Если uint32\_t переменная содержит внутри себя адрес в памяти (является указателем), то перед ее именем должен быть префикс <<p\_>>.
	
	\textbf{Пример:} <<p\_target\_port>>.
	\item <<bit\_banding\_>> == <<bb\_>>
	
	Только в тексте (не применимо к коду).
	\item <<point\_bit\_banding\_bit\_address>> == <<bb\_p\_>>
	
	Когда uint32\_t переменная содержит адрес бита в bit banding области (является указателем).
\end{enumerate}						% Принятые сокращения.
\section{Правила оформления имён}\label{general:rules:0}
\begin{enumerate}
	\item Все имена переменных, структур, объектов, функций должны быть написаны строчными латинскими символами (маленькие английские буквы).
	
	\textbf{Пример: }<<pwr>>, <<port>>, <<value>>.
	\item Директивы препроцессора (define, макросы, ifndef и т.д.) должны писаться заглавными латинскими символами (большие английские буквы). 
	
	\textbf{Пример: }<<ADD(A,B)>>
	\item Слова в именах должны быть разделены нижним подчеркиванием.
	
	\textbf{Пример: }<<buf\_speed>>, <<STM32F2\_\-API\_\-PORT\_\-STM32\_\-F20X\_\-F21X\_\-PORT\_\-STRUCT\_\-CLASS\_\-PIN\_\-H\_>>, <<PORT\_PIN\_0>>
	\item Макросы должны начинаться с префикса <<M\_>>, после чего идет действие, которое он совершает (<<GET>>/<<SET>>).
	
	В именах так же следует использовать принятые сокращения.
	
	\textbf{Пример: }<<M\_GET\_BB\_P\_PER(ADDRESS,BIT)>>
	\item \textbf{Рекомендуется воздержаться от использования enum-ов}.
	
	Заместо них следует использовать \textbf{enum class}.
	\item Имя прототипа enum class должно начинаться с префикса <<EC\_>>. К нему можно обращаться только через <<::>>.
	
	Прямое обращение к значению enum class-а без указания пространства имен - запрещено.
	
	\textbf{Пример: }<<EC\_PORT\_NAME::A>>
\end{enumerate}
							% Правила оформления имён.
\chapter{Оформление .h файлов библиотеки}\label{file:h}
\section{Общее оформление}
\begin{enumerate}
	\item Файл должен включать в себя защиту от повторного включения в процесс компиляции по типу \textit{ifndef-define-endif}, оканчивающуюся пустой строкой.\\\textbf{Пример:}\begin{lstlisting}[language=C++, frame=tlBR, basicstyle=\fontsize{10}{10}\ttfamily]
#ifndef STM32F2_API_STM32_F20X_F21X_PORT_H_
#define STM32F2_API_STM32_F20X_F21X_PORT_H_
	
#endif
 \end{lstlisting}
	
	\item После \textit{define} строки защиты следует пустая строка, за которой располагается \textit{include} на файл конфигурации библиотеки, имеющий имя \textbf{perfix\_conf.h} (название библиотеки выступает в качестве префикса).\\\textbf{Пример:}\begin{lstlisting}[language=C++, frame=tlBR, basicstyle=\fontsize{10}{10}\ttfamily]
#define STM32F2_API_STM32_F20X_F21X_PORT_H_
	
#include "stm32_f20x_f21x_conf.h"\end{lstlisting}
	
	\item В случае, если файл стоит включать в процесс компиляции только при каком-то условии, это условие (обернутое в \textit{ifdef}) необходимо указать через одну пустую строку после \textit{include} файла конфигурации библиотеки. Блок \textit{endif}, закрывающий тело блока условной компиляции должен быть написан без пустой строки перед \textit{endif}, закрывающим блок защиты повторной компиляции.\\\textbf{Пример:}\begin{lstlisting}[language=C++, frame=tlBR, basicstyle=\fontsize{10}{10}\ttfamily]
#ifndef STM32F2_API_STM32_F20X_F21X_PORT_H_
#define STM32F2_API_STM32_F20X_F21X_PORT_H_
	
#include "stm32_f20x_f21x_conf.h"
	
#ifdef MODULE_PORT
	
CODE
	
#endif
#endif
 \end{lstlisting}
	\item Внутри всех необходимых обёрток, описанных выше (на месте слова <<CODE>> примера из предыдущего пункта), располагается основное содержимое, уникальное для каждого типа .h файла:\begin{itemize}
		\item \textit{perfix\_moduleName.h} (дел~\ref{p:modul:h});
		\item \textit{perfix\_moduleName\_constexpr\_func.h} (подраздел~\ref{p:conf:h});
		\item \textit{perfix\-\_moduleName\-\_constexpr\-\_func\-\_class\-\_className.h} (подраздел~\ref{p:conf:ch});
		\item \textit{perfix\_moduleName\_struct.h} и \textit{perfix\-\_moduleName\-\_struct\-\_class\-\_class\-Name.h}\\(подраздел~\ref{p:struc:h});
	\end{itemize}
\end{enumerate}

\section{Содержимое perfix\_moduleName.h файла}\label{p:modul:h}
\begin{enumerate}
	\item include \textit{perfix\_moduleName\_struct.h} и \textit{perfix\_moduleName\_constexpr\_func.h}  файлов, если таковые имеются;
	\item пустая строка;
	\item краткое описание всего модуля, включающее в себя описание всех классов модуля, обернутое в многострочный комментарий с явно обозначенными границами символами <<*>> в количестве 70 штук;
	\item пустая строка;
	\item краткое описание класса, обернутое в много строчный комментарий;
	\item пустая строка;
	\item include \textit{perfix\_moduleName\_struct\_class\_className.h}, если имеется;
	\item пустая строка;
	\item тело класса;
	\item пустая строка;
	\item include файла \textit{perfix\-\_moduleName\-\_constexpr\-\_func\-\_class\-\_className.h}, если имеется;
	\item пустая строка;
	\item пункты 5-12 повторить для всех требуемых классов;
	\item пустая строка.
\end{enumerate}
\textbf{Пример всего файла:}\begin{lstlisting}[language=C++, frame=tlBR, basicstyle=\fontsize{10}{10}\ttfamily]
#ifndef STM32F2_API_STM32_F20X_F21X_PORT_H_
#define STM32F2_API_STM32_F20X_F21X_PORT_H_

#include "stm32_f20x_f21x_conf.h"

#ifdef MODULE_PORT

#include "stm32_f20x_f21x_port_struct.h"						
#include "stm32_f20x_f21x_port_constexpr_func.h"				

/**********************************************************************
 * Краткое описание модуля...
 **********************************************************************/

/*
 * Краткое описание класса pin...
 */

#include "stm32_f20x_f21x_port_struct_class_pin.h"

class pin {
public:
	...

private:
	...
};

#include "stm32_f20x_f21x_port_constexpr_func_class_pin.h"

/*
 * Краткое описание класса global_port...
 */

#include "stm32_f20x_f21x_port_struct_class_global_port.h"		

class global_port {
public:
	...
private:
	...
};

#include "stm32_f20x_f21x_port_constexpr_func_class_global_port.h"

#endif
#endif
 \end{lstlisting}

\section{Содержимое perfix\-\_module\-Name\-\_const\-expr\-\_func.h файла}\label{p:conf:h}
\begin{enumerate}
	\item include \textit{perfix\-\_moduleName\-\_struct.h} файла, если таковой имеются;
	\item пустая строка;
	\item краткое описание constexpr функции, обернутое в много строчный комментарий;
	\item тело функции;
	\item пустая строка;
	\item пункты 3-5 повторить для всех имеющихся методов;
\end{enumerate}
\textbf{Пример всего файла:}\begin{lstlisting}[language=C++, frame=tlBR, basicstyle=\fontsize{10}{10}\ttfamily]
#ifndef STM32F2_API_PORT_STM32_F20X_F21X_PORT_CONSTEXPR_FUNC_H_
#define STM32F2_API_PORT_STM32_F20X_F21X_PORT_CONSTEXPR_FUNC_H_

#include "stm32_f20x_f21x_conf.h"

#ifdef MODULE_PORT

#include "stm32_f20x_f21x_port_struct.h"

/*
 * Краткое описание constexpr функции p_base_port_address_get...
 */

constexpr uint32_t p_base_port_address_get( EC_PORT_NAME port_name ) {
	CODE;
}

/*
 * Краткое описание constexpr функции bb_p_port_look_key_get...
 */
constexpr uint32_t bb_p_port_look_key_get ( EC_PORT_NAME port_name ) {
	CODE;
}

#endif
#endif
 \end{lstlisting}

\section{Содержимое perfix\-\_moduleName\-\_constexpr\-\_func\-\_class\-\_class-\\\-Name.h файла}\label{p:conf:ch}
\begin{enumerate}
	\item include \textit{perfix\_moduleName\_struct.h} и \textit{perfix\_moduleName\_struct\_class\_className.h} файлов, если таковые имеются;
	\item пустая строка;
	\item комментарий о начале области с constexpr конструктором(-ами), обернутый в многострочный комментарий с явно обозначенными границами символами <<*>> в количестве 70 штук. После комментария должна следовать  пустая строка;
	\item тело constexpr конструктора;
	\item пустая строка;
	\item пункты 3-5 повторяются для всех имеющихся конструкторов;
	\item комментарий о начале области с constexpr методами, , обернутый в много строчный комментарий с явно прописанными границами, бросающимися в газа;
	\item пустая строка;
	\item краткое описание constexpr функции, обернутое в много строчный комментарий;
	\item тело функции;
	\item пустая строка;
	\item пункты 9-11 повторить для всех имеющихся методов;
	
\end{enumerate}
\textbf{Пример всего файла:}\begin{lstlisting}[language=C++, frame=tlBR, basicstyle=\fontsize{10}{10}\ttfamily]
#ifndef STM32F2_API_PORT_STM32_F20X_F21X_PORT_CONSTEXPR_FUNC_CLASS_PIN_H_
#define STM32F2_API_PORT_STM32_F20X_F21X_PORT_CONSTEXPR_FUNC_CLASS_PIN_H_

#include "stm32_f20x_f21x_conf.h"

#ifdef MODULE_PORT

#include "stm32_f20x_f21x_port_struct.h"
#include "stm32_f20x_f21x_port_struct_class_pin.h"

/**********************************************************************
 * Область constexpr конструкторов.
 **********************************************************************/
constexpr pin::pin ( const pin_config_t* pin_cfg_array ):
	КОД ИНИЦИАЛИЗАЦИИ ПЕРЕМЕННЫХ КЛАССА {};

/**********************************************************************
 * Область constexpr функций.
 **********************************************************************/
 
/*
 * Краткое описание constexpr функции set_msk_get...
 */
constexpr uint32_t pin::set_msk_get ( const pin_config_t* const pin_cfg_array ) {
	CODE;
}

/*
 * Краткое описание constexpr функции p_base_port_address_get...
 */
constexpr uint32_t pin::p_base_port_address_get ( const pin_config_t* const 
																									pin_cfg_array ) {
	CODE;
}

#endif
#endif
 \end{lstlisting}

\section{Содержимое perfix\_moduleName\_struct.h и\\erfix\_moduleName\_struct\_class\_className.h файлов}\label{p:struc:h}
В файлах располагаются следующие конструкции (сверху вниз):
\begin{enumerate}
	\item Enum class-ы.
	\item Макросы.
	\item Packed структуры.
	\item Структуры.
\end{enumerate}
Правила оформления следующие:
\begin{enumerate}
	\item Перед каждым из выше перечисленных пунктов размещается комментарий о начале области конкретного пункта. Он оборачивается в многострочный комментарий с явно обозначенными границами. Граница представляют из себя строку символов <<*>>. Верхняя и нижняя строка должна содержать 70 разграничивающих символов. После выделенного комментария следует пустая строка.
	\item Перед каждым элементом пункта размещается краткое описание элемента, обернутое в много строчный комментарий. После краткого описания пустая строка не ставится.
	\item Между предыдущим элементом пункта и комментарием следующего элемента этого же пункта вставляется пустая строка.
	\item Между предыдущим элементом пункта и комментария области следующего пункта вставляется пустая строка.
\end{enumerate}
\textbf{Пример всего файла:}\begin{lstlisting}[language=C++, frame=tlBR, basicstyle=\fontsize{10}{10}\ttfamily]
#ifndef STM32F2_API_PORT_STM32_F20X_F21X_PORT_STRUCT_CLASS_PIN_H_
#define STM32F2_API_PORT_STM32_F20X_F21X_PORT_STRUCT_CLASS_PIN_H_

#include "stm32_f20x_f21x_conf.h"

#ifdef MODULE_PORT

/**********************************************************************
 * Область упакованных структур.
 **********************************************************************/
 
/*
 * Краткое описание упакованной структуры.
 */
struct __attribute__((packed)) port_registers_struct {
	ПОЛЯ СТРУКТУРЫ;
};

/**********************************************************************
 * Область структур.
 **********************************************************************/

/*
 * Краткое описание структуры.
 */
struct st_struct {
	ПОЛЯ СТРУКТУРЫ;
};

/**********************************************************************
 * Область enum class-ов.
 **********************************************************************/
 
/*
 * Краткое описание num class-а.
 */
enum class EC_PORT_PIN_NAME {
	VALUE_FIELDS;
};


/**********************************************************************
 * Область макросов.
 **********************************************************************/

/*
 * Краткое описание макроса.
 */
#define M_PIN_CFG_ADC(PORT,PIN)	{
	VALUE_FIELDS;
}

#endif
#endif
 \end{lstlisting}						% Оформление .h файлов библиотеки.
\section{Объявление классов в .h файлах}\label{class:0}
\begin{enumerate}
	\item Общие сведения об оформлении class-ов в .h файлах (подраздел~\ref{OBK}).
	\item В public области должны располагаться (с соблюдением последовательности сверху вниз):\begin{itemize}
		\item конструктор(-ы) класса (подраздел~\ref{K:0:0});
		\item constexpr конструктор(-ы) класса (подраздел~\ref{K:0:1});
		\item constexpr методы класса (подраздел~\ref{constexpr:0});
		\item доступные пользователю нестатические методы класса, выполняющиеся в реальном времени и возвращающие значение стандартного типа или указатель на стандартный тип данных (подраздел~\ref{dp:n:s});
		\item доступные пользователю нестатические методы класса, выполняющиеся в реальном времени и возвращающие значение нестандартного типа или указатель на нестандартный тип данных (подраздел~\ref{dp:n:n};
		\item доступные пользователю статические (static) методы класса, выполняющиеся в реальном времени и возвращающие значение стандартного типа или указатель на стандартный тип (подраздел~\ref{dp:s:s});
		\item доступные пользователю статические (static) методы класса, выполняющиеся в реальном времени и возвращающие значение нестандартного типа или указатель на нестандартный тип (подраздел~\ref{dp:s:n});
		\item открытие переменные и константы класса, доступные пользователю напрямую (подраздел~\ref{dp:op}).
	\end{itemize}
	\item В private область должны располагаться (с соблюдением последовательности сверху вниз):\begin{itemize}
		\item внутренние constexpr методы класса, возвращающие значение стандартного типа или указатель на стандартный тип данных (подраздел~\ref{zp:constexpr:s});
		\item внутренние constexpr методы класса, возвращающие значение нестандартного типа или указатель на нестандартный тип данных (подраздел~\ref{zp:constexpr:n});
		\item закрытые нестатические методы класса, выполняющиеся в реальном времени и возвращающие значение стандартного типа или указатель на стандартный тип данных (подраздел~\ref{zp:n:s});
		\item закрытые нестатические методы класса, выполняющиеся в реальном времени и возвращающие значение нестандартного типа или указатель на нестандартный тип данных (подраздел~\ref{zp:n:n});
		\item закрытые статические (static) методы класса, выполняющиеся в реальном времени и возвращающие значение стандартного типа или указатель на стандартный тип (подраздел~\ref{zp:s:s});
		\item закрытые статические (static) методы класса, выполняющиеся в реальном времени и возвращающие значение нестандартного типа или указатель на нестандартный тип (подраздел~\ref{zp:s:n});
		\item закрытые константы класса стандартных типов (подраздел~\ref{zp:const:s}).
		\item закрытые константы класса нестандартных типов (подраздел~\ref{zp:const:n}).
		\item закрытые переменные класса стандартных типов (подраздел~\ref{zp:pp:s}).
		\item закрытые переменные класса нестандартных типов (подраздел~\ref{zp:pp:n}).
	\end{itemize}
\end{enumerate}

\subsection{Общие сведения об оформлении class-ов в .h файлах}\label{OBK}
\begin{itemize}
	\item Между зарезервированным словом class и именем класса ставится один (1) пробел.
	\item Между последним символом имени класса и открывающейся фигурной скобкой ставится один (1) пробел.
	\item Сначала идет public, а за ним private область.
	\item <<\}>> (скобка закрывающая тело класса) должна находится на новой строке.
\end{itemize}\begin{lstlisting}[language=C++,frame=tlBR]
class name_class {
public:
private:
};\end{lstlisting}

\subsection{Конструктор(-ы) класса}\label{K:0:0}
Использование не constexpr конструкторов классов запрещено. Это связано с неочевидной последовательностью вызова конструкторов глобальных объектов, которая может привести к неверной инициализации объекта (если явно не указывать последовательность вызовов с помощью специальных директив компоновщика). Например, сначала будет предпринята попытка инициализировать внешнюю переферию (за пределами микроконтроллера), не инициализировав интерфейс, по которому она подключена.

В случае если пользователь все же создаст объект, конструктор которого будет требовать выполнения кода функции конструктора во время инициализации, вызов его метода инициализации произведен не будет (объект останется не инициализированным).

\subsection{Constexpr конструктор(-ы) класса}\label{K:0:1}
\begin{itemize}
	\item В случае, если конструкторов несколько, они должны быть расположены от большего количества входных параметров к меньшему.
	\item Реализация самого конструктора не должна находится в теле класса. Ее (реализацию конструктора) следует вынести в отдельный файл.
	\item Перед словом constexpr должен быть выполнен отступ в 1 tab.	
	\item Между словом constexpr и именем конструктора(-ов) ставится один (1) пробел.
	\item После имени конструктора должен быть выполнен один (1) пробел. 
	\item Аргументы конструктора(-ов) в скобках должны быть разделены <<, >> (запятая + пробел).
	\item Внутри скобок перечисления аргументов конструктора должен быть отступ в 1 пробел с каждой стороны.\\\textbf{Пример: } <<( uint32\_t a, uint8\_t b )>>.
\end{itemize}
\textbf{Пример:}\begin{lstlisting}[language=C++, frame=tlBR, basicstyle=\fontsize{8}{8}\ttfamily]
	constexpr pin ( const pin_config_t *pin_cfg_array, const uint32_t pin_cout );
	constexpr pin ( const pin_config_t *pin_cfg_array );
\end{lstlisting}

\subsection{Constexpr методы класса}\label{constexpr:0}
Размещение constexpr методов в разделе public запрещено и не имеет смысла.

\subsection{Доступные пользователю нестатические методы класса, выполняющиеся в реальном времени и возвращающие значение стандартного типа или указатель на стандартный тип данных}\label{dp:n:s}
\begin{itemize}
	\item Перед типом возвращаемого значения должен быть выполнен отступ в один (1) tab.
	\item Имена методов должны быть выравнены с помощью tab с остальными методами этого типа. Выравнивание методов других типов производится по иной сетке.
	\item Аргументы методов в скобках должны быть разделены <<, >> (запятая + пробел).
	\item Внутри скобок перечисления аргументов метода должен быть отступ в один (1) пробел с каждой стороны.\\\textbf{Пример: } <<( uint32\_t a, uint8\_t b )>>.
	\item В случае, если метод не изменяет данные класса, после параметров в скобках следует поставить один (1) пробел, после чего слово <<const;>>. <<;>> закрывает заголовок функции.
\end{itemize}
\textbf{Пример:}\begin{lstlisting}[language=C++, frame=tlBR, basicstyle=\fontsize{8}{8}\ttfamily]
	void	set		( void ) const;
	void	reset		( void ) const;
	void	invert		( void ) const;
	int	read		( void ) const;
\end{lstlisting}

\subsection{Доступные пользователю нестатические методы класса, выполняющиеся в реальном времени и возвращающие значение нестандартного типа или указатель на нестандартный тип данных}\label{dp:n:n}
\begin{itemize}
	\item В качестве нестандартного типа может выступать enum class или структура.
	\item В качестве указателя на нестандартный тип может выступать указатель на enum class переменную или структуру.
	\item Перед возвращаемым типом метода должен быть отступ в один (1) tab.
	\item Имена методов должны быть выравнены с помощью tab с остальными методами этого типа. Выравнивание методов других типов производится по иной сетке.
	\item Аргументы методов в скобках должны быть разделены <<, >> (запятая + пробел).
	\item Внутри скобок перечисления аргументов метода должен быть отступ в 1 пробел с каждой стороны.\\\textbf{Пример: } <<( uint32\_t a, uint8\_t b )>>.
\end{itemize}
\textbf{Пример:}\begin{lstlisting}[language=C++, frame=tlBR, basicstyle=\fontsize{8}{8}\ttfamily]
	E_ANSWER_GP	met_g	( void );
\end{lstlisting}

\subsection{Доступные пользователю статические (static) методы класса, выполняющиеся в реальном времени и возвращающие значение стандартного типа или указатель на стандартный тип}\label{dp:s:s}	\begin{itemize}
	\item Перед зарезервированным словом <<static>> должен быть отступ в один (1) tab.
	\item Между <<static>> и типом возвращаемого значения должен быть выполнен отступ в один (1) пробел.
	\item Имена методов должны быть выравнены с помощью tab с остальными методами этого типа. Выравнивание методов других типов производится по иной сетке.
	\item Аргументы методов в скобках должны быть разделены <<, >> (запятая + пробел).
	\item Внутри скобок перечисления аргументов метода должен быть отступ в 1 пробел с каждой стороны.\\\textbf{Пример: } <<( uint32\_t a, uint8\_t b )>>.
	\item Так как предполагается, что данный метод будет работать с объектом(-ами) класса, в котором(-ых) описан его заголовок, то первым аргументом метода должен быть указатель на void, который внутри класса будет разыменован в указатель на объект класса, в котором был объявлен.\\Используется указатель на void, а не на объект класса с целью совместимости с FreeRTOS, написанной на C.
	\item Первый аргумент метода следует называть <<void *obj>>.
\end{itemize}
\textbf{Пример:}\begin{lstlisting}[language=C++, frame=tlBR, basicstyle=\fontsize{8}{8}\ttfamily]
	static void	task_1	( void *obj );
	static int	m_1	( void *obj );
\end{lstlisting}

\subsection{Доступные пользователю статические (static) методы класса, выполняющиеся в реальном времени и возвращающие значение нестандартного типа или указатель на нестандартный тип}\label{dp:s:n}
\begin{itemize}
	\item В качестве нестандартного типа может выступать enum class или структура.
	\item В качестве указателя на нестандартный тип может выступать указатель на enum class переменную или структуру.
	\item Перед зарезервированным словом <<static>> должен быть отступ в один (1) tab.
	\item Между <<static>> и типом возвращаемого значения должен быть выполнен отступ в один (1) пробел.
	\item Имена методов должны быть выравнены с помощью tab с остальными методами этого типа. Выравнивание методов других типов производится по иной сетке.
	\item Аргументы методов в скобках должны быть разделены <<, >> (запятая + пробел).
	\item Внутри скобок перечисления аргументов метода должен быть отступ в 1 пробел с каждой стороны.\\\textbf{Пример: } <<( uint32\_t a, uint8\_t b )>>.
	\item Так как предполагается, что данный метод будет работать с объектом(-ами) класса, в котором(-ых) описан его заголовок, то первым аргументом метода должен быть указатель на void, который внутри класса будет разыменован в указатель на объект класса, в котором был объявлен.\\Используется указатель на void, а не на объект класса с целью совместимости с FreeRTOS, написанной на C.
	\item Первый аргумент метода следует называть <<void *obj>>.
\end{itemize}
\textbf{Пример:}\begin{lstlisting}[language=C++, frame=tlBR, basicstyle=\fontsize{8}{8}\ttfamily]
	static E_ANSWER_GP	met_a	( void *obj );
\end{lstlisting}

\subsection{Открытие переменные и константы класса, доступные пользователю напрямую}\label{dp:op}
Размещение переменных (изменяемых или заданных как const) в public области запрещено. Даже в случае, если требуется просто читать/записывать одну переменную, следует сделать отдельный метод(-ы) для этого. Так как прямое чтение данных из класса является нарушением ООП.

\subsection{Внутренние constexpr методы класса, возвращающие значение стандартного типа или указатель на стандартный тип данных}\label{zp:constexpr:s}
\begin{itemize}
	\item Реализация тела функции не должна находится в теле класса. Она (реализация тела функции) должна быть вынесена в отдельный файл. Допускаются только заголовки функций.
	\item Перед словом constexpr должен быть выполнен отступ в один (1) tab.
	\item Между зарезервированным словом constexpr и типом возвращаемого значения требуется поставить один (1) пробел.
	\item Имена методов должны быть выравнены с помощью tab с остальными методами этого типа. Выравнивание методов других типов производится по иной сетке.
	\item Аргументы методов в скобках должны быть разделены <<, >> (запятая + пробел).
	\item Внутри скобок перечисления аргументов метода должен быть отступ в 1 пробел с каждой стороны.\\\textbf{Пример: } <<( uint32\_t a, uint8\_t b )>>.
\end{itemize}
\textbf{Пример:}\begin{lstlisting}[language=C++, frame=tlBR, basicstyle=\fontsize{8}{8}\ttfamily]
	constexpr uint32_t	moder_reg_reset_init_msk_get	( EC_PORT_NAME port_name );
\end{lstlisting}

\subsection{Внутренние constexpr методы класса, возвращающие значение нестандартного типа или указатель на нестандартный тип данных}\label{zp:constexpr:n}
\begin{itemize}
	\item В качестве нестандартного типа может выступать enum class или структура.
	\item В качестве указателя на нестандартный тип может выступать указатель на enum class переменную или структуру.
	\item Реализация тела функции не должна находится в теле класса. Она (реализация тела функции) должна быть вынесена в отдельный файл. Допускаются только заголовки функций.
	\item Перед словом constexpr должен быть выполнен отступ в один (1) tab.
	\item Между зарезервированным словом constexpr и типом возвращаемого значения требуется поставить один (1) пробел.
	\item Имена методов должны быть выравнены с помощью tab с остальными методами этого типа. Выравнивание методов других типов производится по иной сетке.
	\item Аргументы методов в скобках должны быть разделены <<, >> (запятая + пробел).
	\item Внутри скобок перечисления аргументов метода должен быть отступ в 1 пробел с каждой стороны.\\\textbf{Пример: } <<( uint32\_t a, uint8\_t b )>>.
\end{itemize}\begin{lstlisting}[language=C++, frame=tlBR, basicstyle=\fontsize{8}{8}\ttfamily]
	constexpr EC_PORT_NAME	port_name_get	( uint32_t value_reg );
\end{lstlisting}

\subsection{Закрытые нестатические методы класса, выполняющиеся в реальном времени и возвращающие значение стандартного типа или указатель на стандартный тип данных}\label{zp:n:s}
Оформляются так же, как и открытые нестатические методы класса, выполняющиеся в реальном времени и возвращающие значение стандартного типа или указатель на стандартный тип данных (подраздел~\ref{dp:n:s}).

\subsection{Закрытые нестатические методы класса, выполняющиеся в реальном времени и возвращающие значение нестандартного типа или указатель на нестандартный тип данных}\label{zp:n:n}
Оформляются так же, как и открытые нестатические методы класса, выполняющиеся в реальном времени и возвращающие значение нестандартного типа или указатель на нестандартный тип данных (подраздел~\ref{dp:n:n}).

\subsection{Закрытые статические (static) методы класса, выполняющиеся в реальном времени и возвращающие значение стандартного типа или указатель на стандартный тип}\label{zp:s:s}
Оформляются так же, как и открытые статические (static) методы класса, выполняющиеся в реальном времени и возвращающие значение стандартного типа или указатель на стандартный тип (подраздел~\ref{dp:s:s}).

\subsection{Закрытые статические (static) методы класса, выполняющиеся в реальном времени и возвращающие значение нестандартного типа или указатель на нестандартный тип}\label{zp:s:n}
Оформляются так же, как и открытые статические (static) методы класса, выполняющиеся в реальном времени и возвращающие значение нестандартного типа или указатель на нестандартный тип (подраздел~\ref{dp:s:n}).

\subsection{Закрытые константы класса стандартных типов}\label{zp:const:s}
\begin{itemize}
	\item На одной строке допустимо объявлять лишь одну константу.
	\item Перед зарезервированным словом <<const>> должен быть поставлен один (1) tab.
	\item Имена констант должны быть выравнены с помощью tab с остальными константами этого типа. Выравнивание констант других типов производится по иной сетке.
\end{itemize}\begin{lstlisting}[language=C++, frame=tlBR, basicstyle=\fontsize{8}{8}\ttfamily]
	const uint32_t	count;
\end{lstlisting}

\subsection{Закрытые константы класса нестандартных типов}\label{zp:const:n}
Оформляются так же, как и закрытые константы класса стандартных типов (подраздел~\ref{zp:const:s})\begin{lstlisting}[language=C++, frame=tlBR, basicstyle=\fontsize{8}{8}\ttfamily]
	const global_port_msk_reg_struct	gb_msk_struct;
\end{lstlisting}

\subsection{Закрытые переменные класса стандартных типов}\label{zp:pp:s}
\begin{itemize}
	\item На одной строке допустимо объявлять лишь одну переменную.
	\item Перед типом должен быть поставлен один (1) tab.
	\item Имена переменных должны быть выравнены с помощью tab с остальными переменными этого типа. Выравнивание переменных других типов производится по иной сетке.
\end{itemize}\begin{lstlisting}[language=C++, frame=tlBR, basicstyle=\fontsize{8}{8}\ttfamily]
	uint32_t	flag;
\end{lstlisting}

\subsection{Закрытые переменные класса нестандартных типов}\label{zp:pp:n}
Оформляются так же, как и закрытые переменные класса стандартных типов (подраздел~\ref{zp:pp:s})\begin{lstlisting}[language=C++, frame=tlBR, basicstyle=\fontsize{8}{8}\ttfamily]
	EC_FL	mb_msk_struct;
\end{lstlisting}						% Объявление классов в .h файлах.
\section{Объявление packed структур}\label{struct:p}
\subsection{Когда стоит объявлять структуру как packed?}
Структуру следует объявить как packed, если она:
\begin{itemize}
	\item описывает совокупность регистров аппаратного блока периферии (у которого, как известно, регистры имеют четко фиксированный размер и порядок следования);
	\item описывает образ памяти регистров аппаратного блока;
	\item описывает структуру какого-либо пакета со строго фиксированными полями.
\end{itemize}

\subsection{Размещение packed структур}
Packed структуры должны быть размещены только в \textbf{.h} файлах. 

\subsection{Оформление packed структур}
\begin{itemize}
	\item перед первой объявленной packed структурой размещается комментарий о начале области packed структур, обернутый в много строчный комментарий с явно обозначенными границами, бросающимися в газа. После чего вставляется пустая строка;
	\item перед каждой packed структурой размещается ее краткое описание, обернутое в много строчный комментарий. После краткого описания пустая строка не ставится;
	\item заголовок структуры следует оформить следующим образом:
	\begin{enumerate}
		\item ключевое слово struct без отступов в начале строки;
		\item отступ в один (1) пробел;
		\item директива препроцессора <<\_\_attribute\_\_((packed))>>;
		\item пробел;
		\item имя структуры;
		\item пробел;
		\item открывающая тело packed структуры скобка <<\{>>;
	\end{enumerate}
	\item поля структуры следует оформлять следующим образом:
	\begin{enumerate}
		\item каждая строка начинается с отступа в один (1) tab;
		\item ключевое слово volatile;
		\item один (1) пробел;
		\item тип поля;
		\item требуемое количество отступов, выполненных с помощью tab;
		\item имя поля;
		\item <<;>>;
		\item требуемое количество tab;
		\item <<// >> (// + пробел) + одно строчный комментарий.
	\end{enumerate}
	\item все имена полей структуры должны быть выравнены с помощью tab между собой;
	\item после последнего поля структуры следует скобка закрытия тела структуры (<<\}>>);
	\item после последней packed структуры вставляется пуста строка.
\end{itemize}\textbf{Пример packed области:}\begin{lstlisting}[language=C++, frame=tlBR, basicstyle=\fontsize{9}{9}\ttfamily]
/**************************************************************
 * Packed struct label.
 **************************************************************/

/*
 * Packed struct description...
 */
struct __attribute__((packed)) port_registers_struct {
	volatile uint32_t	mode;		// Comment...
	volatile uint32_t	otype;		// Comment...
	...;
};

\end{lstlisting}					% Объявление packed структур.
\chapter{Объявление структур}\label{struct}
\section{Когда стоит оборачивать данные в структуру?}
Данные следует обернуть в структуру, если:
\begin{itemize}
	\item требуется передавать более одного параметра в конструктор класса;
	\item требуется вернуть из функции более одного параметра;
\end{itemize}
\textbf{Замечание: }перед тем, как оборачивать данные в структуру проверьте, не имеются ли у вас условий, согласно которым данные должны быть обернуты в упакованную структуру (раздел~\ref{struct:p})
\section{Размещение структур}
Прототипы структур должны быть размещены только в \textbf{.h} файлах. Экземпляры - в \textbf{.cpp}.

\section{Оформление структур}
\begin{itemize}
	\item перед первой объявленной структурой размещается комментарий о начале соответствующей области (области структур), обернутый в многострочный комментарий с явно обозначенными границами символами <<*>> в количестве 70 штук. После комментария должна следовать  пустая строка;
	\item перед каждой структурой размещается ее краткое описание, обернутое в многострочный комментарий. После краткого описания пустая строка не ставится;
	\item заголовок структуры следует оформить следующим образом:
	\begin{enumerate}
		\item ключевое слово struct без отступов в начале строки;
		\item отступ в один (1) пробел;
		\item имя структуры;
		\item пробел;
		\item открывающая тело packed структуры скобка <<\{>>;
	\end{enumerate}
	\item поля структуры следует оформлять следующим образом:
	\begin{enumerate}
		\item каждая строка начинается с отступа в один (1) tab;
		\item тип поля;
		\item требуемое количество отступов, выполненных с помощью tab;
		\item имя поля;
		\item <<;>>;
		\item требуемое количество tab;
		\item <<// >> (// + пробел) + одно строчный комментарий.
	\end{enumerate}
	\item все имена полей структуры должны быть выравнены с помощью tab между собой;
	\item после последнего поля структуры следует скобка закрытия тела структуры (<<\}>>);
	\item после последней структуры вставляется пуста строка.
\end{itemize}\textbf{Пример области структур:}\begin{lstlisting}[language=C++, frame=tlBR, basicstyle=\fontsize{10}{10}\ttfamily]
/**********************************************************************
 * Область структур.
 **********************************************************************/

/*
 * Краткое описание структуры...
 */
struct a {
	uint32_t	b;		// Пояснение к полю b.
	uint32_t	c;		// Пояснение к полю c.
};\end{lstlisting}							% Объявление структур.
\chapter{Объявление enum class-ов}\label{ec:h:0}
\section{Когда стоит объявлять enum class?}
Enum class-ы стоит объявлять, если какому-то полю структуры/переменной требуется присвоить какое-то одно значение из заранее известного ряда, каждому из которых можно присвоить уникальное имя.

\section{Размещение enum class-ов}
Прототипы enum class-ов должны быть размещены только в \textbf{.h} файлах.

\section{Оформление enum class-ов}
\begin{itemize}
	\item перед первым объявленным enum class-ом размещается комментарий о начале соответствующей области (области enum class-ов), обернутый в многострочный комментарий с явно обозначенными границами символами <<*>> в количестве 70 штук. После комментария должна следовать  пустая строка;
	\item перед каждым enum class-ом размещается его краткое описание, обернутое в многострочный комментарий. После краткого описания пустая строка не ставится;
	\item заголовок enum class-а следует оформить следующим образом:
	\begin{enumerate}
		\item ключевое словосочетание <<enum class>> без отступов в начале строки;
		\item отступ в один (1) пробел;
		\item имя enum class-а (в соответствии с правилами написания имен enum class-ов);
		\item пробел;
		\item открывающая тело enum class-а скобка <<\{>>;
	\end{enumerate}
	\item значения enum class-ов следует оформлять следующим образом:
	\begin{enumerate}
		\item каждое значение начинается с отступа в один (1) tab;
		\item имя значения;
		\item требуемое количество отступов, выполненных с помощью tab;
		\item <<=	>> (<<=>> + один (1) tab);
		\item непосредственное цифровое значение;
		\item <<,>> (в случае, если элемент последний, запятая не ставится).
		\item требуемое количество tab;
		\item <<// >> (// + пробел) + однострочный комментарий.
	\end{enumerate}
	\item выравнивание производится по знаку <<=>> с левой стороны;
	\item после последнего значения enum class-а следует скобка закрытия его тела и точка с запятой (<<\};>>), расположенная на новой строке;
	\item после последнего enum class - а вставляется пуста строка.
\end{itemize}\textbf{Пример области структур:}\begin{lstlisting}[language=C++, frame=tlBR, basicstyle=\fontsize{10}{10}\ttfamily]
/**********************************************************************
 * Область enum class- ов.
 **********************************************************************/

enum class EC_PORT_NAME {
	A	=	0,
	B	=	1,
	C	=	2,
	D	=	3,
	H	=	4
};\end{lstlisting}							% Объявление enum class-ов.
\chapter{Объявление макросов}\label{mackros:0}
\section{Когда стоит объявлять макрос?}
Макрос следует объявить, если имеется фрагмент кода, многократно повторяющийся в коде программы с незначительными изменениями в каждом конкретном случае. При этом использовать отдельную constexpr функцию нерационально или невозможно.\\Пример: при заполнении массива структур инициализации выводов некоторое количество выводов инициализируются как входы ADC. Для того, чтобы не писать каждый раз все параметры каждого пина, достаточно будет воспользоваться макросом, в котором нужно указать изменяющиеся параметры: имя порта и вывода.

\section{Размещение макросов}
Прототипы макросов должны быть размещены в \textbf{.h} файлах.

\section{Оформление макросов}
\begin{itemize}
	\item перед первым объявленным макросом размещается комментарий о начале соответствующей области (области макросов), обернутый в многострочный комментарий с явно обозначенными границами символами <<*>> в количестве 70 штук. После комментария должна следовать  пустая строка;
	\item перед каждым макросом размещается его краткое описание, обернутое в многострочный комментарий. После краткого описания пустая строка не ставится;
	\item заголовок макроса следует оформить следующим образом:
	\begin{enumerate}
		\item ключевое слово \#define без отступов в начале строки;
		\item отступ в один (1) пробел;
		\item имя макроса (в соответствии с правилами написания макросов);
		\item открывающаяся скобка (<<(>>);
		\item параметры макроса через запятую без пробелов в соответствии с правилами написания макросов;
		\item закрывающаяся скобка (<<)>>);
		\item тело макроса;
	\end{enumerate}
\end{itemize}\textbf{Пример:}\begin{lstlisting}[language=C++, frame=tlBR, basicstyle=\fontsize{10}{10}\ttfamily]
/**********************************************************************
 * Область макросов.
 **********************************************************************/

/*
 * Возвращает структуру конфигурации вывода,
 * настроенного на вход, подключенный к ADC.
 */
#define M_PIN_CFG_ADC(PORT,PIN)	{											\
	.port							= PORT,														\
	.pin_name					= PIN,														\
	.mode							= EC_PIN_MODE::ANALOG,						\
	.output_config		= EC_PIN_OUTPUT_CFG::NOT_USE,			\
	.speed						= EC_PIN_SPEED::LOW,							\
	.pull							= EC_PIN_PULL::NO,								\
	.af								= EC_PIN_AF::NOT_USE,							\
	.locked						= EC_LOCKED::LOCKED,							\
	.state_after_init	= EC_PIN_STATE_AFTER_INIT::NO_USE	\
}\end{lstlisting}						% Объявление макросов.

\if 0
\fi
\end{document}